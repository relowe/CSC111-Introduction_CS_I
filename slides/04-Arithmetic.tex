\documentclass[]{beamer}
\mode<presentation>
{
  \usetheme{Warsaw}
  \definecolor{mcgarnet}{rgb}{0.38, 0, 0.08}
  \definecolor{mcgray}{rgb}{0.6, 0.6, 0.6}
  \setbeamercolor{structure}{fg=mcgarnet,bg=mcgray}
  %\setbeamercovered{transparent}
}


\usepackage[english]{babel}
\usepackage[latin1]{inputenc}
\usepackage{times}
\usepackage[T1]{fontenc}
\usepackage{tikz}
\usepackage{graphicx}
\usepackage{adjustbox}
\usepackage{fancyvrb}

\newcommand{\imagesource}[1]{{\centering\hfill\break\hbox{\scriptsize Image Source:\thinspace{\small\itshape #1}}\par}}

\title{04 - Arithmetic}


\author{Dr. Robert Lowe\\}

\institute[Maryville College] % (optional, but mostly needed)
{
  Division of Mathematics and Computer Science\\
  Maryville College
}

\date[]{}
\subject{}

\pgfdeclareimage[height=0.5cm]{university-logo}{images/Maryville-College}
\logo{\pgfuseimage{university-logo}}



\AtBeginSection[]
{
  \begin{frame}<beamer>{Outline}
    \tableofcontents[currentsection]
  \end{frame}
}


\begin{document}

\begin{frame}
  \titlepage
\end{frame}

\begin{frame}{Outline}
  \tableofcontents
\end{frame}


% Structuring a talk is a difficult task and the following structure
% may not be suitable. Here are some rules that apply for this
% solution: 

% - Exactly two or three sections (other than the summary).
% - At *most* three subsections per section.
% - Talk about 30s to 2min per frame. So there should be between about
%   15 and 30 frames, all told.

% - A conference audience is likely to know very little of what you
%   are going to talk about. So *simplify*!
% - In a 20min talk, getting the main ideas across is hard
%   enough. Leave out details, even if it means being less precise than
%   you think necessary.
% - If you omit details that are vital to the proof/implementation,
%   just say so once. Everybody will be happy with that.

\section{Types}

\begin{frame}{Variable Types}
    \begin{itemize}
        \item C++ has the following variable types:
            \begin{description}
                \item[\texttt{bool}] Stores a value that is either true or
                    false
                \item[\texttt{char}] Stores a single character (a letter,
                    digit, or any other symbol)
                \item[\texttt{int}] Stores an integer
                \item[\texttt{float}] Stores a single precision floating point
                    number (don't use these!)
                \item[\texttt{double}] Stores a double precision floating point 
                    number.
            \end{description}
        \item Variables must be declared before they are used:
            \newline\texttt{int x;}
            \newline\texttt{char letter;}
            \newline\texttt{double num;}
    \end{itemize}
\end{frame}


\begin{frame}{Literals}
    \begin{itemize}[<+->]
        \item A literal is a value that is typed into a program.
        \item Like variables, literals have types.
        \item The compiler infers literal types from the format of the
            literal.
    \end{itemize}
    
    \uncover<+->{\textbf{Example Literals}}
    \begin{description}[<+->]
        \item[\texttt{bool}] \texttt{true} or \texttt{false}
        \item[\texttt{char}] \texttt{'a'}, \texttt{'b'}, \texttt{'+'}
        \item[\texttt{int}] \texttt{5}, \texttt{10}, \texttt{15}, \texttt{42}
        \item[\texttt{float}] \texttt{1.5f}, \texttt{1.0f}
        \item[\texttt{double}] \texttt{1.5}, \texttt{1.0}
        \item[string] \texttt{"This is a string literal"}
    \end{description}
\end{frame}


\begin{frame}{More About Variable Declarations}
    \begin{itemize}[<+->]
        \item You can declare multiple variables of the same type in
            one statement.
            \newline\texttt{int x, y;}
            \newline\texttt{double a, b;}
        \item Variables can be assigned initial values (initialized)
            during declaration.
            \newline\texttt{int count=0;}
            \newline\texttt{char fi='R', mi='E', li='L';}
    \end{itemize}
\end{frame}


\begin{frame}{Constants}
    \begin{itemize}[<+->]
        \item A constant is like a named literal.
        \item Unlike a variable, a constant cannot be changed. (So
            it's not just a clever name!)
        \item A constant is declared just like a variable, but with
            the \texttt{const} keyword.
        \item The value of a constant must be immediately assigned
            when it is declared.
    \end{itemize}

    \uncover<+->{\textbf{Constant Example}
    \newline\texttt{const double PI=3.14159;}}
\end{frame}

\begin{frame}{Code Style Notes}
    \begin{itemize}[<+->]
        \item Variable names should begin with a lower case letter.
        \item When a variable name has more than one word, either 
            \texttt{camelCase} it or \texttt{use\_underscores}.
        \item Never mix and match camel casing and underscore variable
        naming in the same program!
        \item Constants should be named in all upper case letters.
        \item Always use underscores with constant names with more
            than one word.
        \item Use descriptive variable names, but try to keep it short.
        \item Only use multi-variable declarations where variables are
            related.  For example, this is fine:
            \newline\texttt{double x, y;        //coordinates}
            \newline But this is probably not fine:
            \newline\texttt{int count, length;  //count and length}
    \end{itemize}
\end{frame}

\section{Arithmetic Operators}

\begin{frame}{Integer Arithmetic}
    \begin{description}[<+->]
        \item[$+$] Addition
        \item[$-$] Subtraction
        \item[$*$] Multiplication
        \item[$/$] Division
        \item[$\%$] Modulus (remainder)
    \end{description}

    \begin{itemize}[<+->]
        \item Note that when doing integer arithmetic, C++ truncates
            any fractional parts.
        \item Division works like ``grade school long division''
        \item The operator $/$ simply returns the quotient.
        \item The modulus operator $\%$ returns the remainder of the
            division.
        \item Integer arithmetic is performed on any expression
            consisting of integer literals or variables.
    \end{itemize}
\end{frame}

\begin{frame}{Floating Point Arithmetic}
    \begin{description}[<+->]
        \item[$+$] Addition
        \item[$-$] Subtraction
        \item[$*$] Multipliation
        \item[$/$] Division
    \end{description}

    \begin{itemize}[<+->]
        \item Floating point arithmetic is what a pocket calculator
            typically does.
        \item This deals with real numbers, so they have fractional
            parts.
        \item There is no modulus for floating point arithmetic.
        \item Floating point arithmetic is performed on any expression
            which contains at least one \texttt{double} or
            \texttt{float} literal/variable.
    \end{itemize}
\end{frame}

\begin{frame}{Assignment Operators}
    \begin{description}[<+->]
        \item[\texttt{=}] Assignment
        \item[\texttt{+=}] Addition Assignment
        \item[\texttt{-=}] Subtraction Assignment
        \item[\texttt{*=}] Multiply Assignment
        \item[\texttt{/=}] Divide Assignment
        \item[\texttt{\%=}] Modulus Assignment
    \end{description}

    \begin{itemize}[<+->]
        \item The left hand side of an assignment operator must be
            a variable.
        \item Assignment operators change the value of a variable.
        \item Assignment operators return the value that was assigned.
        \item Compound assignment operators are short-hand ways to
        modify variables.  For example:
        \newline\texttt{x += 1} is short for \texttt{x = x + 1}
    \end{itemize}
\end{frame}

\begin{frame}{Operator Precedence}
    \begin{tabular}{|l|l|l|}
        \hline
        \textbf{Operator} & \textbf{Description} & \textbf{Associativity} \\
        \hline
        \texttt{a*b}, \texttt{a/b}, \texttt{a\%b} & Multiply, Divide, Modulus & Left-to-Right\\
        \hline
        \texttt{a+b}, \texttt{a-b} & Addition and Subtraction & Left-to-Right\\
        \hline
        \texttt{<<} , \texttt{>>} & Insertion nd Extraction & Left-to-Right \\
        \hline
        \texttt{=},  & Assignment and Assignment & Right-to-Left \\
        \texttt{+=}, \texttt{-=} & & \\
        \texttt{*=}, \texttt{/=} & & \\
        \texttt{\%=} & & \\
        \hline
    \end{tabular}

    \begin{itemize}[<+->]
        \item Precedence specifies the order of operations.
        \item Associativity is how we ``break ties''.
        \item Parenthesis can also be used to control order of
            operations (as in normal math).
    \end{itemize}
\end{frame}


\begin{frame}[fragile]{Example: \texttt{pmdas.cpp}}
    \begin{adjustbox}{max height=0.8\textheight}
    \begin{BVerbatim}
#include <iostream>

using namespace std;

int main()
{
    cout << "3+2*6=" << 3+2*6 << endl
         << "(3+2)*6=" << (3+2)*6 << endl
         << "5%2=" << 5%2 << endl
         << "6/2*(1+2)=" << 6/2*(1+2) << endl
         << "1/2*4=" << 1/2*4 << endl
         << "1.0/2.0*4=" << 1.0/2.0*4 << endl;
}
    \end{BVerbatim}
    \end{adjustbox}
\end{frame}

\begin{frame}{Statement Resolution}
    \begin{itemize}[<+->]
        \item C++ Resolves statements via expression substitution.
        \item Once all operations are cleared, the statement is
            completed.
        \item For example:
            \uncover<+->{\newline\texttt{6/2*(1+2)}}
            \uncover<+->{\newline\texttt{6/2*3}}
            \uncover<+->{\newline\texttt{3*3}}
            \uncover<+->{\newline\texttt{9}}
        \item Another example:
            \uncover<+->{\newline\texttt{cout << 2+2 << endl}}
            \uncover<+->{\newline\texttt{cout << 4 << endl}}
            \uncover<+->{\newline\texttt{cout << endl}}
            \uncover<+->{\newline\texttt{cout}}
    \end{itemize}
\end{frame}

\section{Programming With Operators}

\begin{frame}{The Overall Process}
    \begin{enumerate}[<+->]
        \item Write (in English) the steps to perform the program.
        \item Write any formulae needed.
        \item Identify variables and constants.
        \item Decide on variable and constant types.
        \item Start with the boilerplate program.
        \item Declare variables and constants.
        \item Write code to perform the needed operations.
    \end{enumerate}
\end{frame}

\begin{frame}{Lab Activity: Calculate Circle Area (Steps 1-3)}
    \textbf{Problem:} Write a program to calculate the area of
    a circle.

    \uncover<2->{\textbf{1.) Write steps in English}}
    \begin{enumerate}[<+(2)->]
        \item get the radius of the circle
        \item calculate the area
        \item print the results
    \end{enumerate}

    \uncover<+(2)->{\textbf{2.) Write any formulae needed}}
    \uncover<+(2)->{
    \[
        a = \pi r^2
    \]}

    \uncover<+(2)->{\textbf{3.) Identify variables and constants.}}
    \uncover<+(2)->{\newline \texttt{PI, r, a}}

\end{frame}


\begin{frame}[fragile]{Lab Activity: Calculate Circle Area (Steps 4-5)}
    \uncover<+->{\textbf{4.) Decide on variable and constant types}}
    \begin{enumerate}[<+->]
        \item \texttt{ PI: double}
        \item \texttt{ r: double}
        \item \texttt{ a: double}
    \end{enumerate}

    \uncover<+->{\textbf{5.) Start with the boilerplate program}}
    \begin{enumerate}[<+->]
        \item Make your \texttt{labs/week3} directory.
        \item Copy boilerplate.cpp to \texttt{labs/week3/circle.cpp}.
    \end{enumerate}
\end{frame}


\begin{frame}[fragile]{Lab Activity: Calculate Circle Area (Step 6)}
    \textbf{6.) Declare variable and Constants}
    \newline Add the following to he beginning of main:
    \newline
        
    \begin{adjustbox}{max width=0.95\textwidth}
    \begin{BVerbatim}
    const double PI=3.14159;  //the ratio c/d for all circles
    double r;                 //radius of the circle
    double a;                 //The area of the circle
    \end{BVerbatim}
    \end{adjustbox}
\end{frame}

\begin{frame}[fragile]{Lab Activity: Calculate Circle Area (Step 7)}
    \textbf{7.) Write code to perform the needed operations}
    \newline Leave a blank line after the declarations and add the
    following:
    \newline\newline
    \begin{BVerbatim}
    //get the radius of the circle
    cout << "What is the radius of the circle? ";
    cin >> r;

    //calculate the area
    a = PI * r * r;

    //print the results
    cout << "Area: " << a << endl;
    \end{BVerbatim}
    \newline\newline
    Compile and test your program.
\end{frame}

\begin{frame}{Some Notes on Style}
    \begin{itemize}[<+->]
        \item Code within a block should be indented.
        \item Variable declarations should go at the top of a block.
        \item A blank line should follow the variable declarations.
        \item A blank line should separate related chunks of code.
        \item Each chunk of code should have a comment introducing it.
    \end{itemize}
\end{frame}

\begin{frame}[fragile]{A Correctly Formatted \texttt{main} for \texttt{circle.cpp}}
\begin{adjustbox}{max width=0.95\textwidth, max height=0.9\textheight}
\begin{BVerbatim}
int main()
{
    const double PI=3.14159;  //the ratio c/d for all circles
    double r;                 //radius of the circle
    double a;                 //The area of the circle

    //get the radius of the circle
    cout << "What is the radius of the circle? ";
    cin >> r;

    //calculate the area
    a = PI * r * r;

    //print the results
    cout << "Area: " << a << endl;
}
\end{BVerbatim}
\end{adjustbox}
\end{frame}

\begin{frame}[fragile]{Challenge: Additional Circle Calculations}
    \textbf{Challenge:} Add computation of diameter and circumference 
    to your circle program.  
    \newline\newline 
    For example, given a radius of $3$, your program should produce the following output:
    \newline\newline
\begin{BVerbatim}
Diameter: 6
Circumference: 18.8495
Area: 28.2743
\end{BVerbatim}
\end{frame}

\begin{frame}{Lab Activity: Quadratic Equation}
    \textbf{Problem:} Compute the quadratic equation for any set of
    coefficients.
    \uncover<2->{
    \[
        x=\dfrac{-b \pm \sqrt{b^2-4ac}}{2a}
    \]}

    \uncover<3->{Let's carry out the design process.}
\end{frame}

\begin{frame}{The \texttt{cmath} Library}
    \begin{itemize}[<+->]
        \item \texttt{cmath} includes functions like you would find on
            a scientific calculator.
        \item One of these is \texttt{sqrt} which computes the square
            root of a number.
        \item For the rest, please see
        \newline\url{https://en.cppreference.com/w/cpp/header/cmath}.
        \item To use these functions, you must add the following line
            underneath the \texttt{\#include <iostream>}:
        \newline\texttt{\#include <cmath>}
    \end{itemize}
\end{frame}

\begin{frame}[fragile]{Quadratic Equation Variable Declarations}
    Add the following to the appropriate part of your main function.
    \newline\newline
\begin{adjustbox}{max width=0.95\textwidth, max height=0.9\textheight}
\begin{BVerbatim}
    double a, b, c;     //coefficients
    double x1, x2;      //roots
    double rhs;         //right hand side of the numerator
    double divisor;     //the divisor
\end{BVerbatim}
\end{adjustbox}
\end{frame}

\begin{frame}[fragile]{Quadratic Equation: Coefficients}
    Add the following at the appropriate space
    \newline\newline
\begin{BVerbatim}
    //get the coefficients
    cout << "a=";
    cin >> a;
    cout << "b=";
    cin >> b;
    cout << "c=";
    cin >> c;
\end{BVerbatim}
\end{frame}

\begin{frame}[fragile]{Quadratic Equation: Compute and Display}
\begin{adjustbox}{max width=0.95\textwidth, max height=0.9\textheight}
\begin{BVerbatim}
    //compute the right hand side of the numerator
    rhs = sqrt(b*b - 4.0 * a * c);

    //compute the divisor
    divisor = 2.0 * a;

    //compute the roots
    x1 = (-b - rhs) / divisor;
    x2 = (-b + rhs) / divisor;

    //print the results
    cout << "The roots are: " << x1 << ", " << x2 << endl;
\end{BVerbatim}
\end{adjustbox}
\end{frame}

\begin{frame}{Finishing Up}
    \begin{itemize}[<+->]
        \item You should have the following files in \texttt{labs/week3}:
            \begin{itemize}
                \item \texttt{circle.cpp}
                \item \texttt{quadratic.cpp}
            \end{itemize}
        \item Make sure both programs work.
        \item Add, Commit, and Push in git!
    \end{itemize}
\end{frame}

\end{document}


