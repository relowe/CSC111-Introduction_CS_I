\documentclass{beamer}
\mode<presentation>
{
  \usetheme{Warsaw}
  \definecolor{mcgarnet}{rgb}{0.38, 0, 0.08}
  \definecolor{mcgray}{rgb}{0.6, 0.6, 0.6}
  \setbeamercolor{structure}{fg=mcgarnet,bg=mcgray}
  %\setbeamercovered{transparent}
}


\usepackage[english]{babel}
\usepackage[latin1]{inputenc}
\usepackage{times}
\usepackage[T1]{fontenc}
\usepackage{tikz}
\usepackage{graphicx}
\usepackage{xcolor}
\usepackage{adjustbox}
\usepackage{fancyvrb}

\newcommand{\imagesource}[1]{{\centering\hfill\break\hbox{\scriptsize Image Source:\thinspace{\small\itshape #1}}\par}}

\title{03 - Program Structure and Variables}


\author{Dr. Robert Lowe\\}

\institute[Maryville College] % (optional, but mostly needed)
{
  Division of Mathematics and Computer Science\\
  Maryville College
}

\date[]{}
\subject{}

\pgfdeclareimage[height=0.5cm]{university-logo}{images/Maryville-College}
\logo{\pgfuseimage{university-logo}}



\AtBeginSection[]
{
  \begin{frame}<beamer>{Outline}
    \tableofcontents[currentsection]
  \end{frame}
}


\begin{document}

\begin{frame}
  \titlepage
\end{frame}

\begin{frame}{Outline}
  \tableofcontents
\end{frame}


% Structuring a talk is a difficult task and the following structure
% may not be suitable. Here are some rules that apply for this
% solution: 

% - Exactly two or three sections (other than the summary).
% - At *most* three subsections per section.
% - Talk about 30s to 2min per frame. So there should be between about
%   15 and 30 frames, all told.

% - A conference audience is likely to know very little of what you
%   are going to talk about. So *simplify*!
% - In a 20min talk, getting the main ideas across is hard
%   enough. Leave out details, even if it means being less precise than
%   you think necessary.
% - If you omit details that are vital to the proof/implementation,
%   just say so once. Everybody will be happy with that.

\section{Program Structure}

\begin{frame}{Startup}
    \begin{enumerate}
        \item Log in to your shell account
        \item \texttt{cd cs1-fall2019-username}
        \item \texttt{git pull}
    \end{enumerate}
\end{frame}

\begin{frame}[fragile]{\texttt{hello.cpp}}
\begin{tabular}{rl}
 \uncover<2->{\textcolor{green}{\small Preprocessor $\rightarrow$}} & \verb!#include <iostream>!\\
                                     & \\
 & \verb!using namespace std;!\\
 & \\
 \uncover<3->{\textcolor{green}{\small Main Header $\rightarrow$}}  & \verb!int main()!\\
 & \verb!{! \uncover<4->{\textcolor{yellow}{\small $\leftarrow$ Begin Block}}\\
 \uncover<5->{\textcolor{green}{\small Main Body $\rightarrow$}} & \verb!    cout << "hello, world" << endl;!\\
 & \verb!}! \uncover<4->{\textcolor{yellow}{\small $\leftarrow$ End Block}}\\
\end{tabular}
\end{frame}

\begin{frame}{Preprocessor Directives}
    \begin{itemize}[<+->]
        \item The preprocessor runs before the main compilation takes
            place.
        \item Preprocessor directives begin with \texttt{\#}.
        \item The include directive copies the contents of a file to
            its location.
        \item \texttt{iostream} is a C++ library file which contains
            definitions for input and output.
        \item For any program that does input and output in C++, you
            must therefore have the directive:  
            \newline\texttt{\#include<iostream>}
    \end{itemize}
\end{frame}

\begin{frame}{using}
    \begin{itemize}[<+->]
        \item In order to avoid name collisions, C++ has namespaces.
        \item All of the C++ library is in the \texttt{std} namespace.
        \item To access these elements, we would normally have to use
            the \texttt{::} operator.
        \item for example, the cout line in \texttt{hello.cpp} would read:
            \newline\texttt{std::cout << "hello, world" << std::endl;}
        \item Needless to say, this gets tedious!
        \item The using line tells c++ to import all the objects from
            a namespace so we don't have to use \texttt{::} to access
            them.
            \newline
            \texttt{using namespace std;}
    \end{itemize}
\end{frame}

\begin{frame}{The \texttt{main} Function}
    \begin{itemize}[<+->]
        \item Every C++ program begins its execution in the main
            function.
        \item This is formally called \textbf{the program entry point}.
        \item The main function returns an integer to the operating
            system. A 0 means success, all other numbers are errors.
        \item If you do not specify a return value, the compiler will
            default to 0.
        \item All of your code, for now, will go in between the curly
            braces that mark the start and end of the main function.
    \end{itemize}
\end{frame}

\begin{frame}{Blocks and Statements}
    \begin{itemize}[<+->]
        \item A statement is a group of operations terminated by
            a semicolon \texttt{;}.
        \item For example, this line is the only statement in
            \texttt{hello.cpp}:
            \newline
            \texttt{cout << "hello, world" << endl;}
        \item C++ does not care about white space, so a statement can
            span multiple lines.
        \item Groups of statements are called blocks, and they are
            wrapped in curly braces: \{ and \}.
        \item The \texttt{main} function's body is a block of code.
        \item Blocks can be nested inside each other (more on this
            later).
    \end{itemize}
\end{frame}

\begin{frame}[fragile]{Comments}
    \begin{itemize}[<+->]
        \item Comments are notes which explain the code.
        \item The text of comments are ignored by the compiler.
        \item C++ has two types of comments:
            \newline\texttt{// Everything to the end is a comment}
            \newline\texttt{/* Everything within is a comment */}
        \item The \texttt{//} comment is new to C++, and is the
            preferred method.
        \item \texttt{/* */} are c-style comments and can be used to make multi-line
            comments, but be careful!
        \item Every program should have comments (lest they lose
            points when being graded).
    \end{itemize}
\end{frame}

\begin{frame}[fragile]{\texttt{boilerplate.cpp}}
    \begin{enumerate}
        \item \texttt{cd $\sim$/cs1-fall2019-{\em username}}
        \item Create the file \texttt{boilerplate.cpp} and enter the following:
    \end{enumerate}
    \begin{verbatim}
// File:
// Purpose:
// Author:
#include <iostream>

using namespace std;

int main()
{

}
    \end{verbatim}
\end{frame}

\begin{frame}{The Stream Insertion Operator}
    \begin{itemize}[<+->]
        \item \texttt{cout} is the character output stream.
        \item Inserting data into \texttt{cout} will display it on the
            screen.
        \item The operator \texttt{<<} is the \textbf{insertion operator}.
        \item \texttt{endl} is a constant which means ``end of line''.
        \item So the line of C++:
            \newline\texttt{cout << "hello, world" << endl;} 
            \newline means ``Display the words `hello, world' and then
              end the line''
        \item \textbf{Something to Try}: Remove the \texttt{<< endl} portion of
            this line in hello.cpp.  Recompile and run it.  What changed?
    \end{itemize}
\end{frame}


\begin{frame}[fragile]{Multiple Lines of Output}

    Often, we want to have multiple lines of text. This can be done in
    one statement!
    \newline
    \newline
    \begin{adjustbox}{max width=0.95\textwidth}
    \begin{BVerbatim}
cout << "Tell me, where is fancy bred?" << endl
     << "  Or in the heart, or in the head?" << endl
     << "               --William Shakespeare" << endl
     << "                (Merchant of Venice)" << endl;
    \end{BVerbatim}
    \end{adjustbox}
\end{frame}


\begin{frame}[fragile]{Challenge: Draw a Diamond!}
\textbf{Challenge:} Write a program \texttt{diamond.cpp} in your
\texttt{labs/week2} folder which uses a single statement to print the
following figure (begin by copying your \texttt{boilerplate.cpp} file!:
\begin{verbatim}
     #
    ###
   #####
  #######
 #########
###########
 #########
  #######
   #####
    ###
     #
\end{verbatim}
\end{frame}

\section{Variables}
\begin{frame}{The Basic Idea of Variables}
    \begin{itemize}[<+->]
        \item Variables are where programs store data.
        \item Variables can be assigned values, be used in operations,
            and can be changed.
        \item In C++, variables are strongly typed.  That is, each
            variable can only store one type of information!
    \end{itemize}
\end{frame}

\begin{frame}{Types and Declarations}
    \begin{itemize}[<+->]
        \item C++ has the following variable types:
            \begin{description}
                \item[bool] Stores a value that is either true or
                    false
                \item[char] Stores a single character (a letter,
                    digit, or any other symbol)
                \item[int] Stores an integer
                \item[float] Stores a single precision floating point
                    number (don't use these!)
                \item[double] Stores a double precision floating point 
                    number.
            \end{description}
        \item Variables must be declared before they are used:
            \newline\texttt{int x;}
            \newline\texttt{char letter;}
            \newline\texttt{double num;}
    \end{itemize}
\end{frame}

\begin{frame}{Variable Names}

Variables names:
\begin{itemize}[<+->]
    \item must begin with a letter or \_.
    \item can contain letters, numbers, or \_.
    \item are case sensitive.
    \item must be unique.
\end{itemize}
\end{frame}

\begin{frame}{The \texttt{cin} Stream}
    \begin{itemize}[<+->]
        \item \texttt{cin} is the character input stream object.
        \item User input can be read into a variable using the 
            \textbf{extraction operator} \texttt{>>}.
        \item For example:
            \newline\texttt{cin >> x;}
            \newline would allow the user to enter an integer which is
            then stored in \texttt{x}.
    \end{itemize}
\end{frame}

\begin{frame}[fragile]{Example: \texttt{multiple\_choice.cpp}}
Compile and run this program (found in your examples folder)

\begin{adjustbox}{max height=0.75\textheight}
\begin{BVerbatim}
#include <iostream>

using namespace std;

int main()
{
    char choice;  //The choice made by the user

    //Get the user's choice
    cout << "In my opinion, computer programming is _______." << endl
         << "\tA) the best part of my day" << endl
         << "\tB) what gives me a sense of purpose" << endl
         << "\tC) how I scream into the void" << endl
         << endl
         << "Your Choice? ";
    cin >> choice;

    //report the user's choice
    cout << "You chose " << choice << "." << endl;
}
\end{BVerbatim}
\end{adjustbox}
\end{frame}

\section{Stock Portfolio Program}
\begin{frame}{Overview}
Over the course of this class, we will develop an application which
manages a stock portfolio.  It will allow us to:
\begin{itemize}[<+->]
    \item Buy Stocks
    \item Sell Stocks
    \item Run Reports
    \item Store Stock Data in a File
\end{itemize}

{\tiny This program was inspired by a project found in {\em Complete
C Language Programming for the IBM PC} by Douglas A. Troy (1986)}
\end{frame}

\begin{frame}[fragile]{Main Menu}
Write a program \texttt{stock.cpp} which displays the main menu of the
stock portfolio system and reads a user's choice.  
\begin{verbatim}
$ ./stock
        Stock Portfolio Management System
                Please Make a Selection
        1 -- Buy a Stock
        2 -- Sell a Stock
        3 -- Report Current Holdings
        4 -- Report Gains and Losses
        5 -- Remove a Current Holding
        6 -- Done!  (quit) 

        Choice? 6
\end{verbatim}

\end{frame}

\begin{frame}{Finishing Up}
\begin{itemize}
    \item Make sure you have the following files in
        \texttt{cs1-fall2019-{\em username}/labs/week2}
        \begin{itemize}
            \item \texttt{hello.cpp}
            \item \texttt{diamond.cpp}
            \item \texttt{stock.cpp}
        \end{itemize}
    \item Make sure you have \texttt{boilerplate.cpp} in your
        \texttt{cs1-fall2019-{\em username}} directory.
    \item These programs must all be in working order to receive full
        credit for the week!
    \item \texttt{git add -A}
    \item \texttt{git commit -a -m 'Finished Week2!'}
    \item \texttt{git push}
\end{itemize}
\end{frame}

\end{document}


