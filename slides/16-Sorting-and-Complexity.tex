\documentclass[handout]{beamer}
\mode<presentation>
{
  \usetheme{Warsaw}
  \definecolor{mcgarnet}{rgb}{0.38, 0, 0.08}
  \definecolor{mcgray}{rgb}{0.6, 0.6, 0.6}
  \setbeamercolor{structure}{fg=mcgarnet,bg=mcgray}
  %\setbeamercovered{transparent}
}


\usepackage[english]{babel}
\usepackage[latin1]{inputenc}
\usepackage{times}
\usepackage[T1]{fontenc}
\usepackage{tikz}
\usepackage{graphicx}

\newcommand{\imagesource}[1]{{\centering\hfill\break\hbox{\scriptsize Image Source:\thinspace{\small\itshape #1}}\par}}

\title{Sorting Algorithms and Complexity}


\author{Dr. Robert Lowe\\}

\institute[Maryville College] % (optional, but mostly needed)
{
  Division of Mathematics and Computer Science\\
  Maryville College
}

\date[]{}
\subject{}

\pgfdeclareimage[height=0.5cm]{university-logo}{images/Maryville-College}
\logo{\pgfuseimage{university-logo}}



\AtBeginSection[]
{
  \begin{frame}<beamer>{Outline}
    \tableofcontents[currentsection]
  \end{frame}
}


\begin{document}

\begin{frame}
  \titlepage
\end{frame}

\begin{frame}{Outline}
  \tableofcontents
\end{frame}


% Structuring a talk is a difficult task and the following structure
% may not be suitable. Here are some rules that apply for this
% solution: 

% - Exactly two or three sections (other than the summary).
% - At *most* three subsections per section.
% - Talk about 30s to 2min per frame. So there should be between about
%   15 and 30 frames, all told.

% - A conference audience is likely to know very little of what you
%   are going to talk about. So *simplify*!
% - In a 20min talk, getting the main ideas across is hard
%   enough. Leave out details, even if it means being less precise than
%   you think necessary.
% - If you omit details that are vital to the proof/implementation,
%   just say so once. Everybody will be happy with that.

\section{Intuitive Sorting}
\begin{frame}{Activity: Sort your cards}
\end{frame}

\begin{frame}{Activity: How did you sort?}
\end{frame}

\begin{frame}{Activity: Produce Pseudocode of your technique}
\end{frame}

\section{Sorting Algorithms}
\begin{frame}{Sorting Algorithms}
\begin{itemize}[<+->]
    \item Sorting is one of the most commonly studied tasks in
        computer science.
    \item Some of the first algorithms to be studied in terms of
        complexity were sorting algorithms.
    \item Many ``official'' sorting algorithms exist.
    \item These include:
    \begin{itemize}
        \item Selection Sort
        \item Bubble Sort
        \item Merge Sort
    \end{itemize}
\end{itemize}
\end{frame}

\begin{frame}{Selection Sort}
\end{frame}

\begin{frame}{Bubble Sort}
\end{frame}

\begin{frame}{Merge Sort}
\end{frame}

\begin{frame}{Activity: Which of these were you closest to?}
\end{frame}

\section{Time Complexity}
\begin{frame}{Which algorithm is better?}
\end{frame}

\begin{frame}{Asymptotic Notation}
\end{frame}

\begin{frame}{How different can they be?}
\end{frame}

\begin{frame}{Selection Sort Complexity}
\end{frame}

\begin{frame}{Bubble Sort Complexity}
\end{frame}

\begin{frame}{Merge Sort Complexity}
\end{fram}

\begin{frame}{Activity: What is the complexity of your search algorithm?}
\end{frame}

\begin{frame}{Activity: Code a Sorting Algorithm}
\end{frame}

\end{document}


