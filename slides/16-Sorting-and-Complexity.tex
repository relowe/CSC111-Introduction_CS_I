\documentclass[]{beamer}
\mode<presentation>
{
  \usetheme{Warsaw}
  \definecolor{mcgarnet}{rgb}{0.38, 0, 0.08}
  \definecolor{mcgray}{rgb}{0.6, 0.6, 0.6}
  \setbeamercolor{structure}{fg=mcgarnet,bg=mcgray}
  %\setbeamercovered{transparent}
}


\usepackage[english]{babel}
\usepackage[latin1]{inputenc}
\usepackage{times}
\usepackage[T1]{fontenc}
\usepackage{tikz}
\usepackage{graphicx}

\newcommand{\imagesource}[1]{{\centering\hfill\break\hbox{\scriptsize Image Source:\thinspace{\small\itshape #1}}\par}}

\title{Sorting Algorithms and Complexity}


\author{Dr. Robert Lowe\\}

\institute[Maryville College] % (optional, but mostly needed)
{
  Division of Mathematics and Computer Science\\
  Maryville College
}

\date[]{}
\subject{}

\pgfdeclareimage[height=0.5cm]{university-logo}{images/Maryville-College}
\logo{\pgfuseimage{university-logo}}



\AtBeginSection[]
{
  \begin{frame}<beamer>{Outline}
    \tableofcontents[currentsection]
  \end{frame}
}


\begin{document}

\begin{frame}
  \titlepage
\end{frame}

\begin{frame}{Outline}
  \tableofcontents
\end{frame}


% Structuring a talk is a difficult task and the following structure
% may not be suitable. Here are some rules that apply for this
% solution: 

% - Exactly two or three sections (other than the summary).
% - At *most* three subsections per section.
% - Talk about 30s to 2min per frame. So there should be between about
%   15 and 30 frames, all told.

% - A conference audience is likely to know very little of what you
%   are going to talk about. So *simplify*!
% - In a 20min talk, getting the main ideas across is hard
%   enough. Leave out details, even if it means being less precise than
%   you think necessary.
% - If you omit details that are vital to the proof/implementation,
%   just say so once. Everybody will be happy with that.

\section{Intuitive Sorting}
\begin{frame}{Activity: Sort your cards}
\begin{enumerate}
    \item Layout the cards in front of you.
    \item Using some technique which looks at one or two cards at
        once, put the cards into sorted order.
    \item Repeat this a few times, until you are conscious of how you
        do it.
\end{enumerate}
\end{frame}

\begin{frame}{Activity: How did you sort?}
\begin{enumerate}
    \item Think about how you sorted your cards.
    \item Try to write down the general idea about how it was done.
\end{enumerate}
\end{frame}

\begin{frame}{Activity: Produce Pseudocode of your technique}
\begin{enumerate}
    \item Create a directory \texttt{labs/week10}
    \item In this directory, using your favorite editor, create a file
        named \texttt{mysort.txt}
    \item Write pseudocode to describe the sorting method that you
        used.
    \item Compare sorting methods with your neighbors.  Who's sort
        looks better?
\end{enumerate}
\end{frame}

\section{Sorting Algorithms}
\begin{frame}{Sorting Algorithms}
\begin{itemize}[<+->]
    \item Sorting is one of the most commonly studied tasks in
        computer science.
    \item Some of the first algorithms to be studied in terms of
        complexity were sorting algorithms.
    \item Many ``official'' sorting algorithms exist.
    \item These include:
    \begin{itemize}
        \item Selection Sort
        \item Bubble Sort
        \item Merge Sort
    \end{itemize}
\end{itemize}
\end{frame}

\begin{frame}[fragile]{Selection Sort}
\begin{verbatim}
selction_sort(ar)
    for i = 0 to ar
        min=i
        for j = i+1 to ar.size()-1
            if ar[j] < ar[min] 
                min = j
            end if
        end for
        swap ar[i] and ar[j]
    end for
\end{verbatim}
\begin{itemize}[<+->]
    \item Carry out selection sort with your cards.
    \item Intuitively, what is selection sort doing?
\end{itemize}
\end{frame}

\begin{frame}[fragile]{Bubble Sort}
\begin{verbatim}
bubble_sort(ar)
    do
        swapped = false
        for i=0 to ar.size()-2
            if ar[i+1] < ar[i]
                swap ar[i+1] and ar[i]
                swapped = true
            end if
        end for
    while swapped
\end{verbatim}
\begin{itemize}[<+->]
    \item Carry out selection sort with your cards.
    \item Intuitively, what is bubble sort doing?
\end{itemize}
\end{frame}

\begin{frame}[fragile]{Merge Sort}
\begin{verbatim}
merge_sort(ar)
    if ar.size() <= 1 
        retun
    end if

    mid = ar.size() / 2
    merge_sort(ar[0..mid])
    merge_sort(ar[mid+1 .. ar.size()-1) 

merge(left, right)
    While left and right are not empty
        take the smallest of the first element in
        left and right
    end while
\end{verbatim}

\begin{itemize}[<+->]
    \item Carry out merge sort with your cards.
    \item Intuitively, what is merge sort doing?
\end{itemize}
\end{frame}

\begin{frame}{Activity: Compare Sorting Algorithms}
    \begin{enumerate}[<+->]
        \item Which sorting method seemed closer to what you did?
        \item Which sorting method seemed more efficient?
    \end{enumerate}
\end{frame}

\section{Time Complexity}
\begin{frame}{Which algorithm is better?}
\begin{itemize}[<+->]
    \item We can always buy more memory.
    \item We can never buy more time.
    \item We typically evaluate an algorithm based on how long it will
        take to execute.
    \item The standard is to rate the algorithm by the number of steps
        necessary to solve a problem of $n$ size.
\end{itemize}
\end{frame}

\begin{frame}{Asymptotic Notation}
\begin{itemize}[<+->]
    \item Suppose we want to determine the number of steps as
        a function $f(n)$
    \item Usually, we can't find the exact $f(n)$, and even if we
        could,this would not be all that meaningful.
    \item We instead compute an asymptotic bound.
    \item An algorithm is $O(g(n))$ if $f(n) \leq c g(n)$ for some
        constant $c$ for all $n$ above some threshold.
    \item Intuitively, $O(g(n))$ gives us the ``worst case'' run time.
\end{itemize}
\end{frame}

\begin{frame}{How different can they be?}
    \begin{itemize}[<+->]
        \item Some common runtime include:
        \begin{itemize}
            \item Logarithmic  - $O(\mathrm{lg}(n))$
            \item Linear - $O(n)$
            \item n log n - $O(n \mathrm{lg}(n))$
            \item Quadratic - $O(n^2)$
            \item Exponential - $O(2^n)$
        \end{itemize}
        \item Using a spreadsheet, let's see how these functions relate
            to each other.
    \end{itemize}
\end{frame}

\begin{frame}[fragile]{Selection Sort Complexity}
Find the time complexity of the following:
\begin{verbatim}
selction_sort(ar)
    for i = 0 to ar
        min=i
        for j = i+1 to ar.size()-1
            if ar[j] < ar[min] 
                min = j
            end if
        end for
        swap ar[i] and ar[j]
    end for
\end{verbatim}
\end{frame}

\begin{frame}[fragile]{Bubble Sort Complexity}
Find the time complexity of the following:
\begin{verbatim}
bubble_sort(ar)
    do
        swapped = false
        for i=0 to ar.size()-2
            if ar[i+1] < ar[i]
                swap ar[i+1] and ar[i]
                swapped = true
            end if
        end for
    while swapped
\end{verbatim}
\end{frame}

\begin{frame}[fragile]{Merge Sort Complexity}
Find the time complexity of the following:
\begin{verbatim}
merge_sort(ar)
    if ar.size() <= 1 
        retun
    end if

    mid = ar.size() / 2
    merge_sort(ar[0..mid])
    merge_sort(ar[mid+1 .. ar.size()-1) 

merge(left, right)
    While left and right are not empty
        take the smallest of the first element in
        left and right
    end while
\end{verbatim}
\end{frame}

\begin{frame}{Activity: What is the complexity of your search algorithm?}
\begin{enumerate}
    \item Compute the time complexity of your own sorting algorithm.
    \item Include your runtime and justification of your complexity in
        your text file.
\end{enumerate}
\end{frame}

\begin{frame}{Activity: Code a Sorting Algorithm}
\begin{enumerate}
    \item Code your favorite sorting algorithm.
    \item Your program should do the following:
    \begin{enumerate}
        \item Ask for 10 numbers, which it adds to a vector.
        \item Run your sorting algorithm.
        \item Print the sorted vector.
    \end{enumerate}
\end{enumerate}
\end{frame}

\end{document}


