\documentclass{beamer}
\mode<presentation>
{
  \usetheme{Warsaw}
  \definecolor{mcgarnet}{rgb}{0.38, 0, 0.08}
  \definecolor{mcgray}{rgb}{0.6, 0.6, 0.6}
  \setbeamercolor{structure}{fg=mcgarnet,bg=mcgray}
  %\setbeamercovered{transparent}
}


\usepackage[english]{babel}
\usepackage[latin1]{inputenc}
\usepackage{times}
\usepackage[T1]{fontenc}
\usepackage{tikz}
\usepackage{graphicx}

\newcommand{\imagesource}[1]{{\centering\hfill\break\hbox{\scriptsize Image Source:\thinspace{\small\itshape #1}}\par}}

\title{Introduction \& Course Overview}


\author{Dr. Robert Lowe\\}

\institute[Maryville College] % (optional, but mostly needed)
{
  Division of Mathematics and Computer Science\\
  Maryville College
}

\date[]{}
\subject{}

\pgfdeclareimage[height=0.5cm]{university-logo}{images/Maryville-College}
\logo{\pgfuseimage{university-logo}}



\AtBeginSection[]
{
  \begin{frame}<beamer>{Outline}
    \tableofcontents[currentsection]
  \end{frame}
}


\begin{document}

\begin{frame}
  \titlepage
\end{frame}

\begin{frame}{Outline}
  \tableofcontents
\end{frame}


% Structuring a talk is a difficult task and the following structure
% may not be suitable. Here are some rules that apply for this
% solution: 

% - Exactly two or three sections (other than the summary).
% - At *most* three subsections per section.
% - Talk about 30s to 2min per frame. So there should be between about
%   15 and 30 frames, all told.

% - A conference audience is likely to know very little of what you
%   are going to talk about. So *simplify*!
% - In a 20min talk, getting the main ideas across is hard
%   enough. Leave out details, even if it means being less precise than
%   you think necessary.
% - If you omit details that are vital to the proof/implementation,
%   just say so once. Everybody will be happy with that.

\section{Syllabus}
\begin{frame}{Course Description}
An introduction to computer science and structured programming with
emphasis on program design and implementation, debugging,
documentation, and programming projects. Laboratory work supplements
and expands lecture topics and offers supervised practice using
programming. 
\end{frame}

\begin{frame}{Required Materials}
\begin{itemize}
    \item {\em Big C++. 3/e}. Cay Horstmann.
    https://tinyurl.com/bigcpp
    \item An internet connected computer of some sort.
\end{itemize}
\end{frame}

\begin{frame}{Grading}
\begin{tabular}{|l|r|}
\hline
{\bf Category} & {\bf Weight}\\
\hline
Exams & 30\%\\
\hline
Quizzes & 15\% \\
\hline
Attendance \& Hands-On Lab Activities & 20\% \\
\hline
Programming Assignments & 35\%\\
\hline
\end{tabular}
\end{frame}

\begin{frame}{Academic Honesty}
\begin{itemize}[<+->]
    \item Cheating and Plagiarism are Immoral (you are a bad person if
    you do this).
    \item Plagiarism is integrating other people's work into your own
    without attribution.
    \item Cheating is the wholesale copying of another work.
    \item Neither Plagiarism or Cheating will be tolerated in any
    form!
    \item Acceptable Sources of Code:
    \begin{itemize}
        \item Your own original code.
        \item Code from class examples and lab assignments.
        \item Code from your textbook.
    \end{itemize}
    \item An honest failure is respectable.  Cheating is contemptible.
\end{itemize}
\end{frame}

\begin{frame}{Course Policy Overview}
    \begin{itemize}[<+->]
        \item Do not cheat.
        \item Deadlines are set, no extensions will be given.
        \item No late work will be accepted.
        \item Attendance is mandatory.
        \item Excused absences are possible, but do not modify any due
        dates (except in rare circumstances).
        \item Students with disabilities should contact the academic
        support center 981-8124 to coordinate accommodations.
    \end{itemize}
\end{frame}

\section{Linux Environment}

\begin{frame}{Account Setup \& Logging In}
\begin{enumerate}[<+->]
    \item Signup for a free github account: http://github.com 
    \item Visit https://tinyurl.com/cs111-github to share your github
    username and real name with me.
    \item Signup for an MCCS account by visiting
    https://cs.maryvillecollege.edu/signup
    \item Click the link in your email to set your password.
\end{enumerate}
\end{frame}

\begin{frame}{Connecting to MCCS}

    \uncover<+->{
    {\bf Username}: first.last
    {\bf Server}: cs.maryvillecollege.edu
    {\bf Port}: 22}

    \begin{description}[<+->]
        \item[Windows]
            \begin{enumerate}
                \item Launch Putty
                    \includegraphics[width=0.25in]{images/putty}
                \item Enter the servername.
                \item Click open
                \item When prompted, enter your username and then
                password.
            \end{enumerate}
        \hrulefill
        \item[ChromeOS]
            \begin{enumerate}
                \item Install and launch ``Secure Shell''
                \item Fill in connection details and click ``open''
            \end{enumerate}
        \hrulefill
        \item[MacOS X]
            \begin{enumerate}
                \item Launch terminal
                \includegraphics[width=0.25in]{images/macterm}
                \item Enter the command: {\tt ssh
                first.last@cs.maryvillecollege.edu}
                \item Press enter
                \item Enter password when prompted.
            \end{enumerate}
    \end{description}
\end{frame}

\begin{frame}{The Linux Command Prompt}

    \begin{itemize}[<+->]
        \item Commands are entered in the following pattern:
            {\tt command arg1 arg2...}
        \item Pressing ``enter'' executes the commands.
        \item Let's try it out!  Run the following commands:
            \begin{enumerate}
                \item {\tt ls}
                \item {\tt ls -a}
                \item {\tt ls -l -a}
                \item {\tt man ls}
                \item {\tt man man}
            \end{enumerate}
    \end{itemize}
\end{frame}

\begin{frame}{Working With Directories}
\begin{description}[<+->]
    \item[{\tt ls}] List files in the current directory
    \item[{\tt mkdir} {\em name}] Create a Directory
    \item[{\tt pwd}] Show the present working directory
    \item[{\tt cd} {\em directory}] Change Directory
\end{description}

\uncover<+->{Let's try it out!}
\begin{enumerate}[<+->]
    \item {\tt mkdir humor}
    \item {\tt ls}
    \item {\tt cd humor}
    \item {\tt pwd}
\end{enumerate}
\end{frame}

\begin{frame}{Working With Files}
\begin{description}[<+->]
    \item[{\tt cp} {\em src dest}] Copy a file
    \item[{\tt mv} {\em src dest}] Move a File
    \item[{\tt rm} {\em filename}] Remove a file (be careful!)
    \item[{\tt less} {\em filename}] View a file
\end{description}

\uncover<+->{Let's try it out!}
\begin{enumerate}[<+->]
    \item {\tt cp /home/robert.lowe/humor/bees bess}
    \item {\tt ls}
    \item {\tt mv bess bees}
    \item {\tt ls}
    \item {\tt rm bees}
    \item {\tt ls}
    \item {\tt cp ~robert.lowe/humor/* .}
    \item {\tt less threes}
\end{enumerate}
\end{frame}


\begin{frame}{Text Editing}
\begin{itemize}[<+->]
    \item Programming is about entering text into files.
    \item A variety of text editors exist.
    \item People tend to choose friends based on common text editor
    preferences.
    \item The most common editors on our system are:
    \begin{description}
        \item[{\tt nano}] - A simple editor, original designed for
        muggles to use with email.
        \item[{\tt vi}] - Visual Editor a modal text editor.
        \item[{\tt emacs}] - A very elaborate and powerful editor.
    \end{description}
\end{itemize}
\end{frame}

\begin{frame}{Playing Games}
    \uncover<+->{Let's take a look at games installed on the system!}
    \begin{enumerate}[<+->]
       \item {\tt ls /usr/games}
       \item {\tt ls /usr/local/games}
       \item Try typing some of the names that came up.  Some of my
       favorites are:
       \begin{itemize}
         \item {\tt tetris-bsd}
         \item {\tt moon-buggy}
         \item {\tt lander}
         \item {\tt robots}
         \item {\tt fortune}
         \item {\tt adventure}
         \item {\tt zork}
         \item {\tt nethack}
       \end{itemize}
    \end{enumerate}
\end{frame}

\begin{frame}{Homework For Next Time}
    
    Play around with the three major editors and pick which one you
    wish to use. (You can change your mind later.)

    \begin{itemize}
        \item {\bf vi} - Run {\tt vimtutor} from the command prompt to
        get the vim tutorial.
        \item {\bf emacs} - Run {\tt emacs} from the command prompt to
        run emacs.
        \newline press ``Control+h'' followed by ``t'' to get to the
        tutorial
        \item {\bf nano} - Just mess around.  This is not an editor
        for serious programmers, but if you like it then you do you!
    \end{itemize}
\end{frame}
\end{document}


