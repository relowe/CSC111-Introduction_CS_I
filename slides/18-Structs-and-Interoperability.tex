\documentclass[]{beamer}
\mode<presentation>
{
  \usetheme{Warsaw}
  \definecolor{mcgarnet}{rgb}{0.38, 0, 0.08}
  \definecolor{mcgray}{rgb}{0.6, 0.6, 0.6}
  \setbeamercolor{structure}{fg=mcgarnet,bg=mcgray}
  %\setbeamercovered{transparent}
}


\usepackage[english]{babel}
\usepackage[latin1]{inputenc}
\usepackage{times}
\usepackage[T1]{fontenc}
\usepackage{tikz}
\usepackage{graphicx}
\usepackage{fancyvrb}
\usepackage{adjustbox}

\newcommand{\imagesource}[1]{{\centering\hfill\break\hbox{\scriptsize Image Source:\thinspace{\small\itshape #1}}\par}}

\title{18 - Structs and Interoperability}


\author{Dr. Robert Lowe\\}

\institute[Maryville College] % (optional, but mostly needed)
{
  Division of Mathematics and Computer Science\\
  Maryville College
}

\date[]{}
\subject{}

\pgfdeclareimage[height=0.5cm]{university-logo}{images/Maryville-College}
\logo{\pgfuseimage{university-logo}}



\AtBeginSection[]
{
  \begin{frame}<beamer>{Outline}
    \tableofcontents[currentsection]
  \end{frame}
}


\begin{document}

\begin{frame}
  \titlepage
\end{frame}

\begin{frame}{Outline}
  \tableofcontents
\end{frame}


% Structuring a talk is a difficult task and the following structure
% may not be suitable. Here are some rules that apply for this
% solution: 

% - Exactly two or three sections (other than the summary).
% - At *most* three subsections per section.
% - Talk about 30s to 2min per frame. So there should be between about
%   15 and 30 frames, all told.

% - A conference audience is likely to know very little of what you
%   are going to talk about. So *simplify*!
% - In a 20min talk, getting the main ideas across is hard
%   enough. Leave out details, even if it means being less precise than
%   you think necessary.
% - If you omit details that are vital to the proof/implementation,
%   just say so once. Everybody will be happy with that.

\section{Sorting and Searching}
\begin{frame}[fragile]{Sorting}
    \begin{itemize}[<+->]
        \item Change into \texttt{labs/week11}
        \item Open \texttt{buysell.cpp}
        \item Find the following line:
            \newline\verb!//sort(list.begin(), list.end());!
        \item Uncomment the line:
            \newline\verb!sort(list.begin(), list.end());!
        \item Now, try to compile your program.
        \item What does the wall of text that you received tell you?
    \end{itemize}
\end{frame}

\begin{frame}[fragile]{Operators and Functions}
    \begin{itemize}[<+->]
        \item The STL \texttt{sort} method uses the less than operator
            to compare objects.
        \item \texttt{Stock} structures cannot be compared this way.
            Why?
        \item Luckily there is a way we can do this!
        \item First, let's take a look at the general form of this
            operator:
            \newline\verb!lhs < rhs!
        \item We could view this as a function:
            \newline\verb!<(lhs, rhs)!
    \end{itemize}
\end{frame}

\begin{frame}[fragile]{Overloading Operators}
    \begin{itemize}[<+->]
        \item C++ allows us to create \textbf{overloaded operators}.
        \item Essentially, this lets us define operators for our
            custom types, like \texttt{Stock}.
        \item We can define a < operator for \texttt{Stock} like so:
        \begin{adjustbox}{max width=0.9\textwidth}
        \begin{BVerbatim}
//stock comparison
bool operator<(const Stock &lhs, const Stock &rhs)
{
    return lhs.stock_symbol < rhs.stock_symbol;
}
        \end{BVerbatim}
        \end{adjustbox}
        \item Be sure to add a function prototype for your operator in
            the proper section of your file.
        \begin{adjustbox}{max width=0.9\textwidth}
        \begin{BVerbatim}
bool operator<(const Stock &lhs, const Stock &rhs);
        \end{BVerbatim}
        \end{adjustbox}
        \item Now, compile and test your program.
    \end{itemize}
\end{frame}

\begin{frame}{Operator Overloading Rules}
    \begin{itemize}[<+->]
        \item The basic pattern for operator overloading is this:
            \newline \textit{return-type} \texttt{operator}\textit{symbol}( \textit{parameters} )
        \item You cannot overload the following operators:
            \begin{itemize}
                \item \texttt{::}
                \item \texttt{.}
                \item \texttt{.*}
            \end{itemize}
        \item Some operators have special rules (more on this next semester).
        \item At least one parameter to the operator must be a custom
            made type (a \texttt{struct} or a \texttt{class}).
        \item Operators should be made to behave as they would by default.
        \item The idea is to better express intent, so overloaded operators
            should elucidate, not obfuscate.
    \end{itemize}
\end{frame}

\begin{frame}[fragile]{Selling Again}
\begin{adjustbox}{max width=0.9\textwidth}
\begin{BVerbatim}
//sell a stock
void sell(vector<Stock> &list)
{
    //ask the user for the stock
    string stock;
    cout << "Which stock do you want to sell? ";
    cin >> stock;
    
    //find the stock
    auto itr = find(list.begin(), list.end(), stock);

    //if the stock is in the list, remove
    //otherwise print an error message
    if(itr != list.end()) {
        list.erase(itr);
    } else {
        cout << "Could not find stock." << endl;
    }
}
\end{BVerbatim}
\end{adjustbox}
\end{frame}


\begin{frame}[fragile]{The Problem Here}
    \begin{itemize}[<+->]
        \item When we compile this, \texttt{g++} freaks out once more!
        \item Here, the trouble is this line:
            \newline\verb!auto itr = find(list.begin(), list.end(), stock);!
        \item This is a little bit different.  What are the data types
            and operation in use?
        \item The following operator will help us!
        \begin{adjustbox}{max width=0.9\textwidth}
        \begin{BVerbatim}
bool operator==(const Stock &lhs, const string &rhs)
{
    return lhs.stock_symbol == rhs;
}
        \end{BVerbatim}
        \end{adjustbox}
        \item Note the types of the operands!
        \item Also, don't forget the prototype:
        \begin{adjustbox}{max width=0.9\textwidth}
        \begin{BVerbatim}
bool operator==(const Stock &lhs, const string &rhs);
        \end{BVerbatim}
        \end{adjustbox}
    \end{itemize}
\end{frame}


\section{Interacting With Files}
\begin{frame}{Storing Structures in Files}
    \begin{itemize}[<+->]
        \item We need a file format that lets us save and restore
            a \texttt{struct}.
        \item The most common way is to simply put each field on
            a line by itself.
        \item We just have to make sure to read the fields in the same
            order we write them.
        \item Will this work for \texttt{Stock} variables?
    \end{itemize}
\end{frame}

\begin{frame}[fragile]{Saving Stocks}
\begin{adjustbox}{max totalheight=0.9\textheight}
\begin{BVerbatim}
//save the file to disk
void save(vector<Stock> &list) 
{
    //open the file
    ofstream file;
    file.open("STOCK.LST");
    if(not file) {
        //handle error
        cout << "Could not open file for writing." << endl;
        return;
    }

    //write the list to the file
    for(auto itr = list.begin(); itr != list.end(); itr++) {
        file << *itr << endl;
    }

    //close the file
    file.close();
}
\end{BVerbatim}
\end{adjustbox}
\end{frame}


\begin{frame}[fragile]{File Stream Insertion Operator}
    \begin{itemize}[<+->]
        \item We need an insertion operator!
        \item Is there anything special about insertion operators?
        \item NO! 
        \begin{adjustbox}{max width=0.9\textwidth}
        \begin{BVerbatim}
ofstream& operator<<(ofstream &file, const Stock &stock)
{
    file << stock.company_name << endl
         << stock.stock_symbol << endl
         << stock.price << endl
         << stock.quantity << endl;

    return file;
}
        \end{BVerbatim}
        \end{adjustbox}

        \item Don't forget your prototype!
        \begin{adjustbox}{max width=0.9\textwidth}
        \begin{BVerbatim}
ofstream& operator<<(ofstream &file, const Stock &stock);
        \end{BVerbatim}
        \end{adjustbox}
    \end{itemize}
\end{frame}

\begin{frame}[fragile]{Loading Structs}
\begin{adjustbox}{max totalheight=0.9\textheight}
\begin{BVerbatim}
//load the stocks from disk
void load(vector<Stock> &list)
{
    //open the file
    ifstream file;
    file.open("STOCK.LST");
    if(not file) {
        //return if the file does not exist
        return;
    }

    //read to the end of the file
    while(not file.eof()) {
        Stock stock;
        if(file >> stock) {
            //add all successfully read stocks
            list.push_back(stock);
        }
    }

    //close the file
    file.close();
}
\end{BVerbatim}
\end{adjustbox}
\end{frame}

\begin{frame}[fragile]{Extraction Operator}
    \begin{itemize}[<+->]
        \item We need an extraction operator!
        \begin{adjustbox}{max width=0.9\textwidth}
        \begin{BVerbatim}
ifstream& operator>>(ifstream &file, Stock &stock)
{
    file >> stock.company_name
         >> stock.stock_symbol
         >> stock.price
         >> stock.quantity;

    return file;
}
        \end{BVerbatim}
        \end{adjustbox}
        \item And its prototype:
        \begin{adjustbox}{max width=0.9\textwidth}
        \begin{BVerbatim}
ifstream& operator>>(ifstream &file, Stock &stock);
        \end{BVerbatim}
        \end{adjustbox}
        \item And with that, the program should be fully functional
            once more!
    \end{itemize}
\end{frame}

\end{document}


