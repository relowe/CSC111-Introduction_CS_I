\documentclass[]{article}
\usepackage{fullpage}
\usepackage{fancyvrb}
\usepackage{hyperref}
\title{13 - Streams}
\date{}

\setlength{\parindent}{0pt}
\setlength{\parskip}{14pt}

\begin{document}
\maketitle

\section{Streams as Objects}
\begin{itemize}
    \item A stream is an object which allows you to send and receive bytes.
    \item \texttt{cin} is an object of type \texttt{istream}, which is sort for input stream.  (See \url{https://en.cppreference.com/w/cpp/io/basic_istream}) 
    \item \texttt{istream} maintains a buffer of bytes read from whatever source it is attached to.
    \item The state of an \texttt{istream} includes its position within this buffer. 
    \item Several status methods are avaialable:
    \begin{itemize}
        \item \texttt{good()} - True if no error has occured
        \item \texttt{eof()} - True if we have reached the end of file
        \item \texttt{fail()} - True if an error has occured
        \item \texttt{bad()} - True if a non-recoverable error has occurred
    \end{itemize}

    \item On a Unix-like system, we can trigger and end of file event in 
        the standard input by pressing \textit{Ctrl + d} on the keyboard.
\end{itemize}

\pagebreak
\section{Lab Activity: \texttt{average.cpp}}
In your \texttt{labs/week8} directory, create a file called 
\texttt{average.cpp} and enter the following code (without line numbers, of
course!)

\begin{BVerbatim}
     1	//Compute an average of all numbers entered
     2	#include <iostream>
     3	
     4	using namespace std;
     5	
     6	
     7	int main()
     8	{
     9	    double sum = 0.0;  //the sum of all the numbers
    10	    int count=0;       //the number of numbers
    11	
    12	    do {
    13	        //get a number
    14	        int n;
    15	        cout << "Enter a number (Ctrl+d to exit): ";
    16	        cin >> n;
    17	    
    18	        if(not cin.eof()) {
    19	            //add to our sum, if there was a number
    20	            sum += n;
    21	            count++;
    22	        }
    23	    } while(not cin.eof());
    24	
    25	    //print the average
    26	    cout << endl <<endl <<  "The average is: " << sum / count << endl;
    27	}
\end{BVerbatim}

Now, try out a few averages.  It works, so long as you don't misbehave!  But now,
try entering the following:

\begin{BVerbatim}
Enter a number (Ctrl+d to exit): 1
Enter a number (Ctrl+d to exit): 2
Enter a number (Ctrl+d to exit): three
\end{BVerbatim}

The thing goes nuts!  Why?  Well the anwer is in what happens when you
do formatted input.  When you type 1, the stream actually contains:

\begin{tabular}{|c|c|}
    \hline
    \texttt{1} & \texttt{\textbackslash n}  \\
    \hline
\end{tabular}

When the line that reads \texttt{cin >> n} this performs an extraction
operation, which does the following:
\begin{enumerate}
    \item Skip all white space.
    \item Consume characters until the next character does not match
      the type we are extracting.
\end{enumerate}
Because we are extracting an \texttt{int}, the extraction operator will stop
when it encounters the first character which is not a digit. So in this 
example, once extraction is done, the stream will contain:

\begin{tabular}{|c|}
    \hline
    \texttt{\textbackslash n}  \\
    \hline
\end{tabular}

\end{document}
