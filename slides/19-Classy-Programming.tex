\documentclass[]{beamer}
\mode<presentation>
{
  \usetheme{Warsaw}
  \definecolor{mcgarnet}{rgb}{0.38, 0, 0.08}
  \definecolor{mcgray}{rgb}{0.6, 0.6, 0.6}
  \setbeamercolor{structure}{fg=mcgarnet,bg=mcgray}
  %\setbeamercovered{transparent}
}


\usepackage[english]{babel}
\usepackage[latin1]{inputenc}
\usepackage{times}
\usepackage[T1]{fontenc}
\usepackage{tikz}
\usepackage{graphicx}
\usepackage{adjustbox}
\usepackage{fancyvrb}

\newcommand{\imagesource}[1]{{\centering\hfill\break\hbox{\scriptsize Image Source:\thinspace{\small\itshape #1}}\par}}

\title{19 - Classy Programming}


\author{Dr. Robert Lowe\\}

\institute[Maryville College] % (optional, but mostly needed)
{
  Division of Mathematics and Computer Science\\
  Maryville College
}

\date[]{}
\subject{}

\pgfdeclareimage[height=0.5cm]{university-logo}{images/Maryville-College}
\logo{\pgfuseimage{university-logo}}



\AtBeginSection[]
{
  \begin{frame}<beamer>{Outline}
    \tableofcontents[currentsection]
  \end{frame}
}


\begin{document}

\begin{frame}
  \titlepage
\end{frame}

\begin{frame}{Outline}
  \tableofcontents
\end{frame}


% Structuring a talk is a difficult task and the following structure
% may not be suitable. Here are some rules that apply for this
% solution: 

% - Exactly two or three sections (other than the summary).
% - At *most* three subsections per section.
% - Talk about 30s to 2min per frame. So there should be between about
%   15 and 30 frames, all told.

% - A conference audience is likely to know very little of what you
%   are going to talk about. So *simplify*!
% - In a 20min talk, getting the main ideas across is hard
%   enough. Leave out details, even if it means being less precise than
%   you think necessary.
% - If you omit details that are vital to the proof/implementation,
%   just say so once. Everybody will be happy with that.

\section{Objects and Classes}

\begin{frame}{Bundling Variables}
    \begin{itemize}[<+->]
        \item \texttt{struct}s provide us with a way to ``bundle''
            several fields.
        \item This is useful for maintaining \textbf{state} of a composite
            type.
        \item But what if there was another layer of abstraction?
    \end{itemize}
\end{frame}

\begin{frame}{Objects}
    \begin{itemize}[<+->]
        \item An \textbf{object} is an entity with both state and
            behavior.
        \item In addition to fields, objects have their own functions.
        \item The basic idea is to have something in a program that
            both ``remembers'' and ``acts''.
        \item Objects provide a way to model real world entities
            within a program.
    \end{itemize}
\end{frame}

\begin{frame}{Object Example}
    \begin{itemize}[<+->]
        \item Suppose we encountered a soda machine.
        \item What could we do with it?
        \begin{itemize}
            \item Insert Money
            \item Push Buttons
            \item Pull Change Return
        \end{itemize}
        \item What sort of internal state does the machine have?
        \begin{itemize}
            \item Money Deposited
            \item Money in the Vault
            \item Price, Quantity, and Brand of Each Soda
        \end{itemize}
    \end{itemize}
\end{frame}

\begin{frame}{Classes}
    \begin{itemize}[<+->]
        \item Classes define objects.
        \item An object is an instance of a class.
        \item The basic elements of a class are:
        \begin{itemize}[<+->]
            \item Constructors 
            \item Member Variable Declarations ({\em state})
            \item Member Functions ({\em behavior})
        \end{itemize}
        \item Note that sometimes, member variables are called
            \textbf{attributes} and member functions are called
            \textbf{methods}.
    \end{itemize}
\end{frame}

\begin{frame}{Constructors}
    \begin{itemize}[<+->]
        \item A constructor is a block of code that is executed when
            we make an object.
        \item The job of the constructor is to initialize the class.
        \item In C++ a constructor has the same name as its class.
        \item Constructors are declared like functions, but they have
            no return type.
        \item A constructor can take parameters, just like a function.
        \item A class may have multiple constructors (more on this
            later).
    \end{itemize}
\end{frame}

\begin{frame}{Member Variables}
    \begin{itemize}[<+->]
        \item Member variables maintain the state of objects.
        \item Each object has its own set of member variables.
        \item These are somewhat analogous to the fields in a struct.
        \item Member variables should be accessible only to the class's
            code.
    \end{itemize}
\end{frame}

\begin{frame}{Member Functions}
    \begin{itemize}[<+->]
        \item Member functions provide the behavior of objects.
        \item Member functions operate on the member variables of an
            object.
        \item The member variables are always in scope within a member
            function.
    \end{itemize}
\end{frame}

\begin{frame}{Controlling Access}
    \begin{itemize}[<+->]
        \item Classes also allow us to control who has access to the
            members of the class.
        \item In C++, the access levels are:
            \begin{description}
                \item[public] - Public members are accessible
                    by all code.
                \item[private] - Private members are accessible only
                    by code within the class.
                \item[protected] - Protected members are accessible
                    only by code within the class or any subclass
                    (more about this later).
            \end{description}
        \item As a general rule, all member variables should be
            \textbf{private}.
        \item Constructors and member functions should usually be
            \textbf{public}.  
        \item Why do you think this is?
    \end{itemize}
\end{frame}

\section{Classes and Objects in C++}

\begin{frame}[fragile]{Class Definitions in C++}
\begin{BVerbatim}
class Name
{
public:
    //Public Members go here
private:
   //Private Members go here
};
\end{BVerbatim}

\begin{itemize}[<+(1)->]
    \item Class definitions typically go in header files.
    \item They contain function prototypes, constructor prototypes,
        and member variable declarations.
    \item Class names should begin with an upper case letter to set
        them apart from variable names.
    \item Defining a class creates a new type (just like with
        \texttt{struct}.)
\end{itemize}
\end{frame}

\begin{frame}[fragile]{\texttt{examples/19-Classy/soda-machine.h}}
\begin{adjustbox}{max totalheight=0.8\textheight}
\begin{BVerbatim}
#ifndef SODA_H
#define SODA_H

class Soda_Machine
{
public:
    //constructor
    Soda_Machine();

    //deposit money into the machine
    void insert_money(double amount);

    //pull the change return, returning the deposited change
    double change_return();

    //push a button, the return value is any message the machine gives
    //in response
    std::string push_button(int button);

private:
    std::vector<std::string> brand; //brands for the buttons
    std::vector<double> price;      //prices of each soda
    std::vector<int> quantity;      //the quantity of each soda
    double deposit;
    double vault;

    //dispense a soda
    void vend(int slot);
};
#endif
\end{BVerbatim}
\end{adjustbox}
\end{frame}

\begin{frame}[fragile]{Conditional Compilation}
\begin{BVerbatim}
#ifndef SODA_H
#define SODA_H
...
#endif
\end{BVerbatim}

\begin{itemize}[<+->]
    \item Multiple definition of classes (and structs) is not allowed.
    \item A header file may be included more than once.
    \item Wrapping the header contents as above makes the preprocessor
        include the contents only one time.
    \item Always do this with C/C++ header files for safety!
\end{itemize}
\end{frame}

\begin{frame}[fragile]{Implementation of Classes}
\begin{BVerbatim}
//deposit money into the machine
void Soda_Machine::insert_money(double amount)
{
    deposit += amount;
}
\end{BVerbatim}

\begin{itemize}[<+->]
    \item Class methods are typically implemented in a cpp file.
    \item Method names are prefixed with the name of the class and the
        scope resolution modifier.
    \item The same is true of constructors.
    \item Take a look in \texttt{soda-machine.cpp} to see this in
        action.
\end{itemize}
\end{frame}

\begin{frame}[fragile]{Using Objects}
    \begin{itemize}[<+->]
        \item Take a look at \texttt{sodasim.cpp} to see how we use
            a class.
        \item First, we must make an object. This is called
            \textbf{instantiation} of the class.
            \newline\verb!Soda_Machine machine;  //the machine!
        \item We can interact with the object using its member
            functions:
            \newline\verb!machine.insert_money(money);!
        \item Take a look at the rest of the \texttt{main} function to
            see how it interacts with our machine. Isn't the realism
            thrilling?
        \item Compile the program using the following command:
            \newline\verb!g++ sodasim.cpp soda-machine.cpp -o sodasim!
        \item Play with it and see what it does.
    \end{itemize}
\end{frame}

\begin{frame}{Best Practices Recap}
    \begin{itemize}[<+->]
        \item A class name should begin with a capital letter.
        \item Each class should be created in two files:
            \begin{itemize}
                \item A header file (definition)
                \item A cpp file (implementation)
            \end{itemize}
        \item Member variables should be private.
        \item Member methods should usually be public.
        \item Constructors should usually be public.
        \item Always provide a constructor.
    \end{itemize}
\end{frame}

\begin{frame}[fragile]{Finishing the Program}
    \begin{enumerate}
        \item Make the directory \texttt{labs/week12}
        \item From your \texttt{cs1} directory, copy all of the files
            like this:
            \newline\verb!cp examples/19-Classy/* labs/week12!
        \item Implement the rest of the class.  
        \item Implement button pushing in \texttt{sodasim.cpp}
    \end{enumerate}
\end{frame}

\begin{frame}[fragile]{Button Pushing Pseudocode}
    \begin{BVerbatim}
push_button(button) 
    if deposit >= cost of the soda 
        if that soda is sold out
          return "Sold Out"
        else 
          vend the soda
    else if soda is sold out
      return "Sold Out"
    else
      return The brand and the cost of soda
    \end{BVerbatim}
\end{frame}

\begin{frame}[fragile]{Vending Pseudocode}
    \begin{BVerbatim}
vend(soda)
   Subtract 1 from the available quantity of soda
   move the cost of the soda to the vault
   print a message indicating vending and brand
    \end{BVerbatim}
\end{frame}
\end{document}


