\documentclass[]{beamer}
\mode<presentation>
{
  \usetheme{Warsaw}
  \definecolor{mcgarnet}{rgb}{0.38, 0, 0.08}
  \definecolor{mcgray}{rgb}{0.6, 0.6, 0.6}
  \setbeamercolor{structure}{fg=mcgarnet,bg=mcgray}
  %\setbeamercovered{transparent}
}


\usepackage[english]{babel}
\usepackage[latin1]{inputenc}
\usepackage{times}
\usepackage[T1]{fontenc}
\usepackage{tikz}
\usepackage{graphicx}
\usepackage{fancyvrb}
\usepackage{adjustbox}

\newcommand{\imagesource}[1]{{\centering\hfill\break\hbox{\scriptsize Image Source:\thinspace{\small\itshape #1}}\par}}

\title{Structures}


\author{Dr. Robert Lowe\\}

\institute[Maryville College] % (optional, but mostly needed)
{
  Division of Mathematics and Computer Science\\
  Maryville College
}

\date[]{}
\subject{}

\pgfdeclareimage[height=0.5cm]{university-logo}{images/Maryville-College}
\logo{\pgfuseimage{university-logo}}



\AtBeginSection[]
{
  \begin{frame}<beamer>{Outline}
    \tableofcontents[currentsection]
  \end{frame}
}


\begin{document}

\begin{frame}
  \titlepage
\end{frame}

\begin{frame}{Outline}
  \tableofcontents
\end{frame}


% Structuring a talk is a difficult task and the following structure
% may not be suitable. Here are some rules that apply for this
% solution: 

% - Exactly two or three sections (other than the summary).
% - At *most* three subsections per section.
% - Talk about 30s to 2min per frame. So there should be between about
%   15 and 30 frames, all told.

% - A conference audience is likely to know very little of what you
%   are going to talk about. So *simplify*!
% - In a 20min talk, getting the main ideas across is hard
%   enough. Leave out details, even if it means being less precise than
%   you think necessary.
% - If you omit details that are vital to the proof/implementation,
%   just say so once. Everybody will be happy with that.

\section{Aggregate Data Types}
\begin{frame}{The Problem}
    \begin{itemize}[<+->]
        \item We often need to store multiple related pieces of
            information.
        \item For instance, what does a stock holding look like?
        \begin{itemize}
            \item Company Name
            \item Stock Symbol
            \item Purchase Price
            \item Quantity
        \end{itemize}
        \item How could we store this for each stock?
    \end{itemize}
\end{frame}

\begin{frame}[fragile]{One Solution: Parallel Vectors}
    \begin{itemize}[<+->]
        \item We could make a vector for each field:
        \begin{verbatim}
vector<string> company_name;
vector<string> stock_symbol;
vector<double> price;
vector<int> quantity;
        \end{verbatim}
        \item This way, stock \texttt{i} has \texttt{company\_name[i]}, 
            \texttt{stock\_symbol[i]}, \texttt{price[i]}, and
            \texttt{quantity[i]}
        \item This is, of course, error prone and awkward!
    \end{itemize}
\end{frame}

\begin{frame}[fragile]{A Better Approach}
    \begin{itemize}
        \item Wouldn't it be nicer if we could do something like this?
        \begin{verbatim}
Stock s;
        \end{verbatim}
        \item Or even better, make a vector of stocks?
        \begin{verbatim}
vector<Stock> portfolio;
        \end{verbatim}
        \item It turns out, we can!
    \end{itemize}
\end{frame}

\begin{frame}[fragile]{C++ \texttt{struct}}

\begin{block}{Stock Structure}
{\tt
struct Stock\{
\newline\verb!    !string company\_name;
\newline\verb!    !string stock\_symbol;
\newline\verb!    !double price;
\newline\verb!    !int quantity;
\newline\};
}
\end{block}
\begin{itemize}[<+(2)->]
    \item A \texttt{struct} is a programmer defined \textbf{aggregate type}.
    \item A \texttt{struct} creates a custom type, which we can then use as any
        other variable type.
    \item The items within a \texttt{struct} are called fields.
\end{itemize}

\end{frame}

\section{Programming With Structures}

\begin{frame}[fragile]{Declaring and Using Struct Variables}
\begin{itemize}[<+->]
    \item Structs operate like any other variable type, and fields are
        accessed using the '.' operator.
        \begin{verbatim}
Stock s;
s.company_name = "Microsoft";
s.stock_symbol = "MSFT";
        \end{verbatim}
    \item The same is true of structs in vectors:
    \begin{verbatim}
vector<Stock> list(10);
list[0].company_name = "Microsoft";
list[0].stock_symbol = "MSFT";
    \end{verbatim}

\end{itemize}
\end{frame}

\begin{frame}[fragile]{Defining Structures}
    \begin{itemize}[<+->]
        \item Typically, a \texttt{struct} will be defined in the
            global scope.
        \item \texttt{struct} definitions should go before function
            prototypes (either in .h or .cpp files).
        \item A typical layout of a .h file is:
        \begin{enumerate}
            \item \texttt{struct} definitions.
            \item Function Prototypes
        \end{enumerate}
        \item A typical layout for a .cpp file is:
        \begin{enumerate}
            \item Includes
            \item \texttt{struct} definitions
            \item Function Prototypes
            \item Main Function
            \item Function Definitions
        \end{enumerate}
    \end{itemize}
\end{frame}

\begin{frame}{\textbf{Lab Activity}: buysell.cpp}
    \begin{enumerate}[<+->]
        \item Create the directory \texttt{labs/week11}.
        \item Copy \texttt{buysell.cpp} from \texttt{labs/week10} to
            \texttt{labs/week11}.
        \item We will be modifying this file to use structures.
        \item Make the following changes to buysell.cpp
    \end{enumerate}
\end{frame}


\begin{frame}[fragile]{buysell.cpp Includes and Structure}
\begin{adjustbox}{max totalheight=0.90\textheight}
\begin{BVerbatim}
//Buy and sell stocks by symbol
#include <iostream>
#include <vector>
#include <string>
#include <algorithm>
#include <fstream>
#include <iomanip>

using namespace std;

//type definitions
struct Stock{
    string company_name;
    string stock_symbol;
    double price;
    int quantity;
};
\end{BVerbatim}
\end{adjustbox}
\end{frame}


\begin{frame}[fragile]{buysell.cpp Function Prototypes}
\begin{adjustbox}{max totalheight=0.90\textheight}
\begin{BVerbatim}
//function prototypes
void buy(vector<Stock> &list);
void sell(vector<Stock> &list);
void display(vector<Stock> &list);
void load(vector<Stock> &list);
void save(vector<Stock> &list);
\end{BVerbatim}
\end{adjustbox}
\end{frame}


\begin{frame}[fragile]{buysell.cpp Main Function}
\begin{adjustbox}{max totalheight=0.90\textheight}
\begin{BVerbatim}
int main()
{
    int choice;
    enum menu_choices { BUY=1, SELL, DISPLAY, QUIT };
    vector<Stock> stocks;  

    //load the stocks
    load(stocks);
    
    //display the menu
    do {
        //get the user's choice
        cout << "       MENU " << endl
             << "   1.) Buy a Stock" << endl
             << "   2.) Sell a Stock" << endl
             << "   3.) Display Stocks" << endl
             << "   4.) Quit" << endl
             << endl
             << "   Choice? ";
        cin >> choice;

        //do the menu choice
        if(choice == BUY) {
            buy(stocks);
        } else if(choice == SELL) {
            sell(stocks);
        } else if(choice == DISPLAY) {
            display(stocks);
        } else if(choice != QUIT) {
            cout << "Invalid selection. Please try again." << endl;
        }
    } while(choice != QUIT);

    //save the stocks
    save(stocks);
}
\end{BVerbatim}
\end{adjustbox}
\end{frame}


\begin{frame}[fragile]{buysell.cpp Buy}
\begin{adjustbox}{max totalheight=0.90\textheight}
\begin{BVerbatim}
//buy a stock
void buy(vector<Stock> &list)
{
    // Ask the user for a stock
    Stock stock;

    //get the stock Fields
    cout << "Company Name> ";
    cin >> stock.company_name;
    cout << "Symbol> ";
    cin >> stock.stock_symbol;
    cout << "Price> ";
    cin >>  stock.price;
    cout << "Quantity> ";
    cin >> stock.quantity;

    // add the stock to the list
    list.push_back(stock);

    //sort the stocks
    //sort(list.begin(), list.end());
}
\end{BVerbatim}
\end{adjustbox}
\end{frame}


\begin{frame}[fragile]{buysell.cpp Sell}
\begin{adjustbox}{max totalheight=0.90\textheight}
\begin{BVerbatim}
//sell a stock
void sell(vector<Stock> &list)
{
/*
    //ask the user for the stock
    string stock;
    cout << "Which stock do you want to sell? ";
    cin >> stock;
    
    //find the stock
    auto itr = find(list.begin(), list.end(), stock);

    //if the stock is in the list, remove
    //otherwise print an error message
    if(itr != list.end()) {
        list.erase(itr);
    } else {
        cout << "Could not find stock." << endl;
    }
*/
}
\end{BVerbatim}
\end{adjustbox}
\end{frame}


\begin{frame}[fragile]{buysell.cpp Display}
\begin{adjustbox}{max totalheight=0.90\textheight}
\begin{BVerbatim}
//display our stocks
void display(vector<Stock> &list)
{
    //print a header
    cout << left << setw(6) << "Symbol" 
         << left << setw(15) << "Company Name" 
         << right << setw(10) << "Quantity"
         << right << setw(10) << "Price" << endl
         << setfill('=') << setw(41) << '=' << endl
         << setfill(' ');
    //loop over all the stocks
    for(auto itr = list.begin(); itr != list.end(); itr++) { 
        cout << left << setw(6) << itr->stock_symbol
             << left << setw(15) << itr->company_name
             << right << setw(10) << itr->quantity 
             << right << fixed << setprecision(2) << setw(10)
             << itr->price 
             << endl;
    }
}
\end{BVerbatim}
\end{adjustbox}
\end{frame}

\begin{frame}[fragile]{buysell.cpp Load}
\begin{adjustbox}{max totalheight=0.90\textheight}
\begin{BVerbatim}
//load the stocks from disk
void load(vector<Stock> &list)
{
/*
    //open the file
    ifstream file;
    file.open("STOCK.LST");
    if(not file) {
        //return if the file does not exist
        return;
    }

    //read to the end of the file
    while(not file.eof()) {
        string stock;
        if(file >> stock) {
            //add all successfully read stocks
            list.push_back(stock);
        }
    }

    //close the file
    file.close();
*/
}
\end{BVerbatim}
\end{adjustbox}
\end{frame}


\begin{frame}[fragile]{buysell.cpp Save}
\begin{adjustbox}{max totalheight=0.90\textheight}
\begin{BVerbatim}
//save the file to disk
void save(vector<Stock> &list) 
{
/*
    //open the file
    ofstream file;
    file.open("STOCK.LST");
    if(not file) {
        //handle error
        cout << "Could not open file for writing." << endl;
        return;
    }

    //write the list to the file
    for(auto itr = list.begin(); itr != list.end(); itr++) {
        file << *itr << endl;
    }

    //close the file
    file.close();
    */
}
\end{BVerbatim}
\end{adjustbox}
\end{frame}
\end{document}
