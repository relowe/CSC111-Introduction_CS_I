\documentclass[]{beamer}
\mode<presentation>
{
  \usetheme{Warsaw}
  \definecolor{mcgarnet}{rgb}{0.38, 0, 0.08}
  \definecolor{mcgray}{rgb}{0.6, 0.6, 0.6}
  \setbeamercolor{structure}{fg=mcgarnet,bg=mcgray}
  %\setbeamercovered{transparent}
}


\usepackage[english]{babel}
\usepackage[latin1]{inputenc}
\usepackage{times}
\usepackage[T1]{fontenc}
\usepackage{tikz}
\usepackage{graphicx}
\usepackage{fancyvrb}
\usepackage{adjustbox}

\newcommand{\imagesource}[1]{{\centering\hfill\break\hbox{\scriptsize Image Source:\thinspace{\small\itshape #1}}\par}}

\title{11 - Functions and Topdown Design}


\author{Dr. Robert Lowe\\}

\institute[Maryville College] % (optional, but mostly needed)
{
  Division of Mathematics and Computer Science\\
  Maryville College
}

\date[]{}
\subject{}

\pgfdeclareimage[height=0.5cm]{university-logo}{images/Maryville-College}
\logo{\pgfuseimage{university-logo}}



\AtBeginSection[]
{
  \begin{frame}<beamer>{Outline}
    \tableofcontents[currentsection]
  \end{frame}
}


\begin{document}

\begin{frame}
  \titlepage
\end{frame}

\begin{frame}{Outline}
  \tableofcontents
\end{frame}


% Structuring a talk is a difficult task and the following structure
% may not be suitable. Here are some rules that apply for this
% solution: 

% - Exactly two or three sections (other than the summary).
% - At *most* three subsections per section.
% - Talk about 30s to 2min per frame. So there should be between about
%   15 and 30 frames, all told.

% - A conference audience is likely to know very little of what you
%   are going to talk about. So *simplify*!
% - In a 20min talk, getting the main ideas across is hard
%   enough. Leave out details, even if it means being less precise than
%   you think necessary.
% - If you omit details that are vital to the proof/implementation,
%   just say so once. Everybody will be happy with that.

\section{Finishing the Romans}
\begin{frame}{Roman Numeral Conversion Functions}
    After decomposing the problem, one possible set of functions
    for convert Indian numbers to Roman Numerals is:
    \begin{itemize}[<+(1)->]
        \item \texttt{print\_roman\_numeral(value)}
        \item \texttt{indian\_to\_roman(num)}
        \item \texttt{repeat\_roman(n, value)}
        \item \texttt{next\_roman(value)}
    \end{itemize}
\end{frame}

\begin{frame}{\texttt{repeat\_roman(n, value)}}
    The design for \texttt{repeat\_roman} goes something like this:
    \begin{enumerate}[<+(2)->]
        \item Call \texttt{print\_roman\_numeral(value)} \texttt{n}
            times.
    \end{enumerate}
\end{frame}

\begin{frame}{\texttt{next\_roman(value)}}
    \begin{itemize}[<+(1)->]
        \item Of course, the hard part is going through the possible
            Roman values.
        \item This function gives us the next number in the sequence:
            1000, 900, 500, 400, 100, 90, 50, 40, 10, 9, 5, 4, 1
        \item We should make sure that if we give it a 1, we return 1.
            Why?
        \item We could implement this using a big chain of if .. else if 
            statements.
        \item Can we come up with a more clever way to pull off the
            function?
    \end{itemize}
\end{frame}

\begin{frame}{Roman Recap}
    \begin{itemize}[<+(1)->]
        \item Roman numeral conversions would have been a nightmare
            if we tried to do this all in one function!
        \item Thanks to the ability to subdivide the problem into
            different functions, it was only a slightly uncomfortable
            dream.
        \item Modular decomposition is the crucial skill in any large
            scale program.
        \item By our simple ape-brained standards, almost all programs
            are large scale problems.
    \end{itemize}
\end{frame}


\section{Scope}
\begin{frame}{Scope}
    \begin{block}{Scope}<2->
        A \textbf{scope} is a region of program text.  Identifiers are
        declared within a scope and are only available within their
        scope.
    \end{block}
    \begin{itemize}[<+(2)->]
        \item Identifiers within a scope must be unique.
        \item The C++ scopes are:
        \begin{itemize}
            \item \textbf{global scope} The entire program text.
            \item \textbf{namespace scope} A named scope (more about
                this much later).
            \item \textbf{class scope} The region within a class.
                (more about this later this semester).
            \item \textbf{local scope} The region within a function
                body or the function's argument list.
            \item \textbf{block/statement scope} The region within
                a statement or between two curly braces \{ \}.
        \end{itemize}
    \end{itemize}
\end{frame}

\begin{frame}[fragile]{Scope Nesting: \texttt{11-Functions/scope.cpp}}
\begin{adjustbox}{max totalheight=0.9\textheight, max width=0.95\textwidth}
\begin{BVerbatim}
#include<iostream>

using namespace std;

//Global Scope

void count(int start, int stop, int increment)
{
    for(int i=start; i<=stop; i+=increment) {
        cout << i << endl;
    }
}

int main()
{
    int start, stop, increment;

    //read in start stop and increment
    cout << "Enter start stop and increment: ";
    cin >> start >> stop >> increment;

    //count
    count(start, stop, increment);
}
\end{BVerbatim}
\end{adjustbox}
\end{frame}


\begin{frame}[fragile]{A Scope Puzzle: \texttt{11-Functions/puzzle.cpp}}
\begin{adjustbox}{max totalheight=0.9\textheight, max width=0.95\textwidth}
\begin{BVerbatim}
// What will the following program display?
#include<iostream>

using namespace std;

void scope_test(int x)
{
    x += 10;
}


int main()
{
    int x = 32;

    scope_test(x);

    cout << x << endl;
}
\end{BVerbatim}
\end{adjustbox}
\end{frame}


\begin{frame}{Argument Passing in C++}
    \begin{itemize}[<+(2)->]
        \item In C++, the default behavior for arguments is
            \textbf{pass by value}.
        \begin{itemize}
            \item The value of the argument is copied into the local
                scoped variable named in the parameter list.
            \item In \texttt{puzzle.cpp}, even though they have the
                same name, \texttt{x} in \texttt{main} is a different
                variable than \texttt{x} in \texttt{scope\_test}.
            \item The two \texttt{x}'s are in a different scope.
        \end{itemize}
        \item We can also \textbf{pass by reference}.
        \begin{itemize}
            \item A reference parameter is declared by placing an
                \texttt{\&} between the type and parameter name.
            \item For example: \texttt{scope\_test(int \&x)}
            \item This binds the parameter to the argument variable, 
                so that both names refer to the same actual variable.
        \end{itemize}
    \end{itemize}
\end{frame}


\begin{frame}[fragile]{Another Scope Puzzle: \texttt{11-Functions/ref.cpp}}
\begin{adjustbox}{max totalheight=0.9\textheight, max width=0.95\textwidth}
\begin{BVerbatim}
// We can also pass by reference!  What will this display?
#include<iostream>

using namespace std;

void scope_test(int &x)
{
    x += 10;
}


int main()
{
    int x = 32;

    scope_test(x);

    cout << x << endl;
}
\end{BVerbatim}
\end{adjustbox}
\end{frame}



\section{Code Reuse and Multi-File Programming}
\begin{frame}{Why Reuse Code?}
    \begin{itemize}[<+(1)->]
        \item Software is complex and expensive to produce.
        \item If we had to constantly rewrite everything, we would 
            never be able to get any work done.
        \item We need some way to separate generic code from a 
            specific application.
        \item This allows us to reuse code in future projects!
    \end{itemize}
\end{frame}

\begin{frame}{Working With Multiple Files}
    \begin{itemize}[<+->]
        \item One easy way to reuse function is to put them into 
            separate files.
        \item Often, we put a main function into a file by itself,
            and then the functions that it calls go into one or
            more additional files.
        \item In this way, we can copy files between programs, or
            even write several programs that use the same functions in
            the same directory.
    \end{itemize}
\end{frame}

\begin{frame}[fragile]{Separating \texttt{roman.cpp}}
    \begin{enumerate}[<+->]
        \item Create the \texttt{labs/week7} directory.
        \item Copy \texttt{labs/week6/roman.cpp} to
            \texttt{labs/week7}
        \item Now, create a new file \texttt{roman-converter.cpp} and 
            move your main function over to this file. (Be sure to
            delete your old main function and to include all the
            iostream stuff that it needs.)
        \item Try to compile your program:
            \newline{\footnotesize\verb!g++ -o roman-converter roman.cpp roman-converter.cpp!}
        \item Did it work?  Why or why not?
    \end{enumerate}
\end{frame}

\begin{frame}{Gluing it Together With Header Files}
    \begin{itemize}[<+->]
        \item A function must be declared before it can be used.
        \item Because the definitions are in a separate file, they are
            not declared in the file that contains our main function.
        \item We can solve this with prototypes.
        \item Repeating prototypes in every file is painful.
        \item Enter the header file!  
        \item A header file usually has a \texttt{.h} extension and
            contains:
            \begin{itemize}
                \item Function Prototypes
                \item Constants
                \item Type Definitions
            \end{itemize}
        \item Let's create \texttt{roman.h}
    \end{itemize}
\end{frame}

\begin{frame}[fragile]{\texttt{roman.h}}
The contents of \texttt{roman.h}:
\begin{BVerbatim}
void indian_to_roman(int num);
\end{BVerbatim}
\begin{itemize}[<+(1)->]
    \item We only include prototypes for functions we want to call
        from outside of this file.
    \item This is the public interface for the roman module.
    \item The other functions are sometimes referred to as
        \textbf{helper functions}.
    \item Technically, what we have does not conform to best
        practices, but we will omit some other details for now.
    \item Aren't I a merciful fellow?
\end{itemize}
\end{frame}

\begin{frame}[fragile]{Including Header Files}
    \begin{itemize}[<+->]
        \item Header files are included using the \verb!#include!
            preprocessor directive.
        \item When you use angle brackets: \verb!< >!, the
            preprocessor searches the system's include directories for the
            file.
        \item When you use quotes: \verb!" "!, the preprocessor
            searches the local directory for the file.
        \item So when you do \verb!#include<iostream>!, you are
            searching for the system file called \texttt{iostream}.
        \item To include your header file, change the top of
        \texttt{roman-converter.cpp} to read as follows:
        \begin{BVerbatim}
#include <iostream>
#include "roman.h"
        \end{BVerbatim}
        \item Now we can compile this multi-part program:
            \newline{\footnotesize\verb!g++ -o roman-converter roman.cpp roman-converter.cpp!}
    \end{itemize}
\end{frame}

\begin{frame}[fragile]{Some Best Practices Regarding Header Files}
    \begin{itemize}[<+->]
        \item Always name your headers with the \texttt{.h} extension.
        \item This makes them distinguishable from the C++ system
            library.
        \item When including files, start with the system-wide
            includes, and then list your own:
            \begin{BVerbatim}
    #include <iostream>
    #include <cmath>
    #include <time.h>
    #include "fun_functions.h"
    #include "easy_functions.h"
    #include "awful_functions.h"
            \end{BVerbatim}
        \item Keep all of your include directives together at the top
            of your files.
        \item Header files may include other files, but do not put
            a \texttt{using namespace} in a header file.  
    \end{itemize}
\end{frame}

\end{document}
