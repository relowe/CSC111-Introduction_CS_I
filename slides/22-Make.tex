\documentclass[]{beamer}
\mode<presentation>
{
  \usetheme{Warsaw}
  \definecolor{mcgarnet}{rgb}{0.38, 0, 0.08}
  \definecolor{mcgray}{rgb}{0.6, 0.6, 0.6}
  \setbeamercolor{structure}{fg=mcgarnet,bg=mcgray}
  %\setbeamercovered{transparent}
}


\usepackage[english]{babel}
\usepackage[latin1]{inputenc}
\usepackage{times}
\usepackage[T1]{fontenc}
\usepackage{tikz}
\usepackage{graphicx}
\usepackage{fancyvrb}
\usepackage{adjustbox}

\newcommand{\imagesource}[1]{{\centering\hfill\break\hbox{\scriptsize Image Source:\thinspace{\small\itshape #1}}\par}}

\title{Make -- Scripting Multi-File Builds}


\author{Dr. Robert Lowe\\}

\institute[Maryville College] % (optional, but mostly needed)
{
  Division of Mathematics and Computer Science\\
  Maryville College
}

\date[]{}
\subject{}

\pgfdeclareimage[height=0.5cm]{university-logo}{images/Maryville-College}
\logo{\pgfuseimage{university-logo}}



\AtBeginSection[]
{
  \begin{frame}<beamer>{Outline}
    \tableofcontents[currentsection]
  \end{frame}
}


\begin{document}

\begin{frame}
  \titlepage
\end{frame}


% Structuring a talk is a difficult task and the following structure
% may not be suitable. Here are some rules that apply for this
% solution: 

% - Exactly two or three sections (other than the summary).
% - At *most* three subsections per section.
% - Talk about 30s to 2min per frame. So there should be between about
%   15 and 30 frames, all told.

% - A conference audience is likely to know very little of what you
%   are going to talk about. So *simplify*!
% - In a 20min talk, getting the main ideas across is hard
%   enough. Leave out details, even if it means being less precise than
%   you think necessary.
% - If you omit details that are vital to the proof/implementation,
%   just say so once. Everybody will be happy with that.

\begin{frame}{The Problem}
\begin{itemize}[<+->]
    \item Most software applications require dozens, or even hundreds,
        of source files!
    \item The building of applications requires many complex commands.
    \item This can become very unwieldy very quickly.
    \item We solve this by using some sort of build system.
\end{itemize}
\end{frame}


\begin{frame}{Multi-Stage Compilation}
\begin{itemize}[<+->]
    \item Compiling the entire source every time is quite
    time consuming.
    \item Instead we split the compilation into two parts:
    \begin{enumerate}
        \item Compile cpp files.
        \item Link cpp files together.
    \end{enumerate}
    \item We can do this by adding the \texttt{-c} option to
        \texttt{g++}
\end{itemize}
\end{frame}

\begin{frame}{Let's Try It!}
\begin{enumerate}[<+->]
    \item Change into the \texttt{examples/19-Classy} directory.
    \item Compile each of the cpp files like this:
    \begin{itemize}
        \item \texttt{g++ -c soda-machine.cpp}
        \item \texttt{g++ -c sodasim.cpp}
    \end{itemize}
    \item List the directory.  Note the \textbf{object files} this
        created.
    \item Now we will link the executable with the following:
        \newline\texttt{g++ soda-machine.o sodasim.o -o sodasim}
\end{enumerate}
\end{frame}

\begin{frame}{Enter Make}
\begin{itemize}[<+->]
    \item Linking object files is faster than compiling source files.
    \item We only need to recompile the object files when the source
        file changes.
    \item This is still a heavy workload!
    \item This where the tool \texttt{make} comes in.
    \item \texttt{make} lets us script the build process in an
        intelligent way.
    \item \texttt{make} works by processing ``recipes''.
    \item Recipes are either implicit or explicitly.
\end{itemize}
\end{frame}


\begin{frame}{Implicit Recipes}
\begin{itemize}[<+->]
    \item Make is ``smart enough'' to build some things without extra
        input.
    \item For instance, try the following:
    \begin{enumerate}
        \item Change into your \texttt{examples/01-Intro-C++}
            directory
        \item Run the command: \texttt{rm hello} to erase the old
            binary (if you have one)
        \item Type \texttt{make hello} and press enter.
        \item Now run \texttt{make hello} again.
    \end{enumerate}
    \item This is make's implicit compilation rules.  It is smart
        enough to convert a single file program into an executable.
    \item One more thing to try:
        \newline\texttt{make hello.o}
    \item Pretty neat, huh?
\end{itemize}
\end{frame}


\begin{frame}[fragile]{Makefile -- Explicit Recipes}
\begin{itemize}[<+->]
    \item When we compile multiple files, we need to explicitly tell
        make how to go about doing it.
    \item We do this by creating a file named ``\texttt{Makefile}''
    \item In the \texttt{Makefile} we write a series of
        \textbf{recipes} in the following format:
        \newline\texttt{target: ingredient list}
    \item For example, try the following:
    \begin{enumerate}
        \item Change to the \texttt{examples/19-Classy} directory.
        \item Create a file named \texttt{Makefile} with the following
        content:
\begin{adjustbox}{max width=\textwidth, max totalheight=0.9\textheight}
\begin{BVerbatim}
sodasim: sodasim.o soda-machine.o
    g++ -o sodasim sodasim.o soda-machine.o

sodasim.o: sodasim.cpp soda-machine.h
soda-machine.o: soda-machine.cpp soda-machine.h
\end{BVerbatim}
\end{adjustbox}
    \end{enumerate}
    \item Remember that when indenting, you must use a literal tab
        character!
\end{itemize}
\end{frame}


\begin{frame}[fragile]{Some Predefined Variables}
\begin{itemize}[<+->]
    \item The make syntax is itself a scripting language.
    \item Variables begin with dollar signs \$.  
    \item There are several pre-defined variables, the two most
        commonly used ones are:
        \begin{itemize}
            \item \verb!$@! -- The name of the target
            \item \verb!$^! -- The list of all ingredients
        \end{itemize}
    \item We could simplify the sodasim \texttt{Makefile} like so:
\begin{adjustbox}{max width=\textwidth, max totalheight=0.9\textheight}
\begin{BVerbatim}
sodasim: sodasim.o soda-machine.o
    g++ -o $@ $^

sodasim.o: sodasim.cpp soda-machine.h
soda-machine.o: soda-machine.cpp soda-machine.h
\end{BVerbatim}
\end{adjustbox}
\end{itemize}
\end{frame}


\begin{frame}{User Defined Variables}
\begin{itemize}[<+->]
    \item You can also define your own variables:
        \newline \texttt{TARGETS=stock}
    \item You refer to your own variables like this:
        \newline  \texttt{\$(TARGETS)}
    \item This allows you to make compact makefiles.
\end{itemize}
\end{frame}

\begin{frame}[fragile]{Making The Program 5 Makefile}
\begin{adjustbox}{max width=\textwidth, max totalheight=0.9\textheight}
\begin{BVerbatim}
TARGETS=stock

#application builds
all: $(TARGETS)
stock: iofun.o main.o stock.o transaction.o portfolio.o
	g++ -o $@ $^

#object files
iofun.o: iofun.h iofun.cpp
main.o: main.cpp iofun.h stock.h transaction.h portfolio.h
stock.o: stock.h stock.cpp
transction.o: transaction.cpp transaction.h
portfolio.o: portfolio.cpp portfolio.h


#delete all binaries
clean:
	rm -f *.o $(TARGETS)
\end{BVerbatim}
\end{adjustbox}
\end{frame}

\begin{frame}{Building With Make}
    \begin{itemize}[<+->]
        \item Run \texttt{make} to build the first recipe in the
            \texttt{Makefile}
        \item Run \texttt{make target} to build any other target.
        \item For example \texttt{make clean} runs the clean target.
    \end{itemize}
\end{frame}

\begin{frame}[fragile]{Lab Activity -- Address Book}
    Using the Program 5 makefile as an example, build a makefile for
    the address book lab.
\begin{adjustbox}{max width=\textwidth, max totalheight=0.9\textheight}
\begin{BVerbatim}
TARGETS=stock

#application builds
all: $(TARGETS)
stock: iofun.o main.o stock.o transaction.o portfolio.o
	g++ -o $@ $^

#object files
iofun.o: iofun.h iofun.cpp
main.o: main.cpp iofun.h stock.h transaction.h portfolio.h
stock.o: stock.h stock.cpp
transction.o: transaction.cpp transaction.h
portfolio.o: portfolio.cpp portfolio.h


#delete all binaries
clean:
	rm -f *.o $(TARGETS)
\end{BVerbatim}
\end{adjustbox}
\end{frame}
\end{document}


